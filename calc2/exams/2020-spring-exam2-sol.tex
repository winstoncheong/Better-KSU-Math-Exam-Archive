\documentclass[12pt,answers]{exam}

\usepackage{amsmath,amsfonts,amssymb,mathtools,physics,commath}
\usepackage{todonotes}
\usepackage{float}
\usepackage{multicol}
\usepackage{polynom}

\newcommand{\inv}{^{-1}}

\pagestyle{headandfoot}
\firstpageheadrule
\runningheadrule
\firstpageheader{Math 221}{Exam 2|Solutions, Page \thepage\ of \numpages}{March 3, 2020}
\runningheader{Math 221}{Exam 2|Solutions, Page \thepage\ of \numpages}{March 3, 2020}
\runningfooter{}{}{}

\begin{document}
% \maketitle
\begin{questions}
	\question
	Evaluate the following integrals.
	\begin{parts}
		\part[10]
		$\displaystyle \int \frac{x^3+2x+1}{x^2+4} \dif x$
		\begin{solution}
			Long division gives
			\[
				\polylongdiv{x^3+2x+1}{x^2+4}
			\]
			so
			\begin{align*}
				\int \frac{x^3+2x+1}{x^2+4} \dif x
				&= \int \left( x + \frac{-2x+1}{x^2+4}\right) \dif x \\ 
				&= \frac12 x^2 - \int \frac{2x}{x^2+4} \dif x + \int \frac{1}{x^2+4} \dif x \\ 
				&= \boxed{\frac12 x^2 - \ln(x^2+4) + \frac12 \tan[-1](\frac x2) + C}
			\end{align*}
		\end{solution}
		\part[12]
		$\displaystyle \int \frac{3x+5}{(x^2+2x+1)(x+2)} \dif x$
		\begin{solution}
			Partial Fractions:
			\[
				\frac{3x+5}{(x+1)^2(x+2)} = \frac{A}{x+2} + \frac{B}{x+1} + \frac{C}{(x+1)^2}
			\]
			Clearing denominators:
			\[
				3x+5 = A(x+1)^2 + B(x+2)(x+1) + C(x+2)
			\]
			Solving gives $A = -1$, $B = 1$, $C = 2$.
			Hence
			\begin{align*}
				\int \frac{3x+5}{(x^2+2x+1)(x+2)} \dif x
				&= \int \left( \frac{-1}{x+2} + \frac{1}{x+1} + \frac{2}{(x+1)^2}\right) \\ 
				&= \boxed{-\ln|x+2| + \ln|x+1| - \frac{2}{x+1} + C}
			\end{align*}
			
		\end{solution}
	\end{parts}

\newpage
\question
Approximate the definite integral $\int_{-4}^4 \sqrt{16-x^2} \dif x$ using
\begin{parts}
	\part[8]
	The Midpoint rule for $M_4$. (Do not simplify the arithmetic.)
	\begin{solution}
		$\Delta x = \frac{4-(-4)}{4} = \frac84 = 2$. The values are $x_i = -4, -2, 0, 2, 4$.
		The midpoints are $m_i = -3, -1, 1, 3$.
		Midpoint rule says
		\begin{align*}
			M_4 
			&= \Delta x \left( f(-3) + f(-1) + f(1) + f(3) \right) \\ 
			&= \boxed{2 \cdot \left(\sqrt 7 + \sqrt{15} + \sqrt{15} + \sqrt 7\right)}
		\end{align*}
	\end{solution}
	\part[8]
	Simpson's rule for $S_4$. (Do not simplify the arithmetic.)
	\begin{solution}
		$\Delta x = \frac{4-(-4)}{4} = \frac84 = 2$. The values are $x_i = -4, -2, 0, 2, 4$.
		Simpson's rule says
		\begin{align*}
			S_4 
			&= \frac{\Delta x}{3} \left( f(-4) + 4f(-2) + 2f(0) + 4f(2) + f(4) \right) \\ 
			&= \boxed{\frac23 \left( 0 + 4\sqrt{12} + 2\sqrt{16} + 4\sqrt{12} + 0\right) } \\ 
			&= \frac23 \left( 0 + 8\sqrt{3} + 8 + 8\sqrt{3} + 0\right) 
		\end{align*}
	\end{solution}
\end{parts}

% \newpage
\question[8]
A spring requires a force of 4 newtons to stretch 2 meters beyond its rest length. How much work is required to stretch the spring from 2 meters to 4 meters beyond its rest length?
\begin{solution}
	The first sentence allows us to determine the spring constant $k$:
	\[
		F = kx 
		\implies 4 = k (2)
		\implies k = 2
	\]
	Then
	\[
		W = \int_2^4 2x \dif x = \eval{x^2}_2^4 = 16 - 4 = \boxed{12 \text{ Joules}}
	\]
\end{solution}

\newpage
\question Evaluate the following improper integrals, or state that they do not exist. Use proper limit notation.
\begin{parts}
	\part[6]
	$\displaystyle \int_2^5 \frac{\dif x}{\sqrt{x-2}}$
	\begin{solution}
		\begin{align*}
			\displaystyle \int_2^5 \frac{\dif x}{\sqrt{x-2}}
			 = \lim_{b\to 2^{-}} \int_b^5 \frac{\dif x}{\sqrt{x-2}}
			 & = \lim_{b\to 2^{-}} \left[ 2 \eval{\sqrt{x-2}}_b^5 \right] \\ 
			 & = \lim_{b\to 2^{-}} 2 \left( \sqrt 3 - \sqrt{b-2} \right) \\
			 & = 2\sqrt 3 - 0 = \boxed{2 \sqrt 3}
		\end{align*}
	\end{solution}
	\part[6]
	$\displaystyle \int_3^\infty \frac{\dif x}{(x-2)^3}$
	\begin{solution}
		\begin{align*}
			\int_3^\infty \frac{\dif x}{(x-2)^3}
			= \lim_{b\to\infty} \int_3^b \frac{\dif x}{(x-2)^3}
			&= \lim_{b\to\infty} \left[ \eval{\frac{(x-2)^{-2}}{-2}}_3^b \right] \\ 
			&= -\frac12 \lim_{b\to\infty} \left( \frac{1}{(b-2)^2} - \frac{1}{(3-2)^2} \right)\\
			&= -\frac12 (0 - 1) 
			= \boxed{\frac12}
		\end{align*}
	\end{solution}
	% \newpage
	\part[4]
	$\displaystyle \int_{-2}^2 \frac{\dif x}{x^2}$
	\begin{solution}
		\begin{align*}
		\int_{-2}^2 \frac{\dif x}{x^2}
		&= \lim_{a\to 0^-} \int_{-2}^a x^{-2} \dif x + \lim_{b\to 0^+} \int_b^2 x^{-2} \dif x \\
		&= \lim_{a\to 0^-} \left[ \eval{-x^{-1}}_{-2}^a \right]
		+ \lim_{b\to 0^+} \left[ \eval{-x^{-1}}_b^2 \right] \\ 
		&= - \lim_{a\to 0^-} \left[ \frac{1}{a} - \frac{1}{-2} \right]
		- \lim_{b\to 0^+} \left[ \frac12 - \frac 1b \right] \\ 
		&= - \left(-\infty + \frac12\right)
		- \lim_{b\to 0^+} \left[ \frac12 - \frac 1b \right]
		\end{align*}
		The integral \fbox{diverges / does not exist}
	\end{solution}
\end{parts}

\newpage
\question
\begin{parts}
	\part[8]
	Find the arc length of the curve $y = \sin x$, $0 \le x \le \frac\pi2$. Just set up the integral. \textbf{Do not evaluate.}
	\begin{solution}
		\[
			L 
			= \boxed{\int_0^{\pi/2} \sqrt{1+ \cos^2 x} \dif x}
		\]
	\end{solution}
	\part[10]
	Find the surface area of the surface generated by rotating  the curve in part (a) around the $x$-axis. \textbf{Evaluate the integral.}
	Make use of an appropriate integral formula on the cover page.
	\begin{solution}
		\begin{align*}
			SA 
			&= \int_0^{\pi/2} 2\pi \sin x \sqrt{1+\cos^2 x} \dif x \qquad (u=\cos x; \dif u = -\sin x \dif x) \\
			&= -2\pi \int_1^{0} \sqrt{1+u^2} \dif u \\
			&= 2\pi \int_0^{1} \sqrt{u^2+1} \dif u \\
			&= 2\pi \left[\frac12 \left(u \sqrt{u^2+1} + \ln\abs{u+\sqrt{u^2+1}}\right)\right]_0^1 \quad (\text{formula on cover page})\\ 
			&= \pi \left( \sqrt 2 + \ln\abs{1+\sqrt 2} - (0 + \ln\abs{1}) \right) \\
			&= \boxed{\pi \left( \sqrt 2 + \ln(1+\sqrt 2) \right)}
		\end{align*}
	\end{solution}
\end{parts}

\newpage
\question[10]
How much work is done by winding up a hanging cable of length 50 feet and weight density 2 lb/ft.
\begin{solution}
Each segment of cable of length $\Delta x$ has force $2\Delta x$ lbs. A segment at height $x_i \in [0, 50]$ will need to be displaced $50-x_i$. The total work is thus 
\begin{align*}
	W &= \sum_{x_i} (50 - x_i) 2 \Delta x \\
	&\xrightarrow{N \to \infty}
	\int_0^{50} (50-x) 2 \dif x \\
	&= \eval{100x - x^2}_0^{50} 
	= \boxed{2500 \text{ ft-lbs}}
\end{align*}
\end{solution}
\question[10]
Find the centroid $(\overline x, \overline y)$ of the region bounded by the semicircle $y = \sqrt{4-x^2}$, $-2 \le x \le 2$ and the $x$-axis. 
(You may use the area formula for a circle, and symmetry to determine one of the values $\overline x, \overline y$.)
\begin{solution}
	Due to symmetry of the region, we know $\overline{x} = 0$.
	To compute $\overline y$: 
	\begin{align*}
		m &= \int_{-2}^2 \sqrt{4-x^2} \dif x = \frac12 \pi \cdot 4 = 2\pi \qquad \text{(area formula)}\\ 
		M_x &= \frac 12 \int_{-2}^2 (4 - x^2) \dif x \qquad\quad \text{(symmetry of even functions)}\\
		&= \frac 12 \cdot 2 \int_0^2 (4-x^2) \dif x \\ 
		&= \eval{4x-\frac{x^3}{3}}_0^2 = 8 - \frac 83 = \frac{16}{3}
		\\
		\overline y &= \frac{M_x}{m} = \frac{16}{3} \cdot \frac{1}{2\pi} = \frac{8}{3\pi}
	\end{align*}
	Thus the centroid is $\boxed{(0, \tfrac{8}{3\pi})}$
\end{solution}
\end{questions}
\end{document}