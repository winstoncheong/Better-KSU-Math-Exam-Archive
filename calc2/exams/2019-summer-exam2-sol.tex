\documentclass[12pt,answers]{exam}

\usepackage{amsmath,amsfonts,amssymb,mathtools,physics,commath}
\usepackage{todonotes}
\usepackage{float}
\usepackage{multicol}
\usepackage{fancybox}
\usepackage{siunitx}
\usepackage{cancel}

\newcommand{\inv}{^{-1}}
\newcommand{\RR}{\mathbb{R}}

\pagestyle{headandfoot}
\firstpageheadrule
\runningheadrule
\firstpageheader{Math 221}{Exam 2|Solutions, Page \thepage\ of \numpages}{2019 Summer}
\runningheader{Math 221}{Exam 2|Solutions, Page \thepage\ of \numpages}{2019 Summer}
\runningfooter{}{}{}

\begin{document}
% \maketitle
\begin{questions}

\question[10]
Evaluate the series $\displaystyle \sum_{n=2}^\infty \left(\frac45\right)^n$
\begin{solution}
  This is a geometric series. It converges to the value
  \[
    \frac{(\frac45)^2}{1-\frac45} = \frac{16}{25} \cdot 5 = \boxed{\frac{16}{5}}
  \]
\end{solution}

\question[10]
Find the minimum $M$ that guarantees that
\[
  \left|
  \sum_{n=1}^\infty \frac{(-1)^{n+1}}{10n+2} -
  \sum_{n=1}^M \frac{(-1)^{n+1}}{10n+2} -
  \right| < .001.
\]
\begin{solution}
  \begin{align*}
    \abs{\frac{(-1)^{M+2}}{10(M+1) + 2}} & < \frac{1}{1000}        \\
    \implies 10(M+1)+2                   & > 1000                  \\
    \implies M+1                         & > \frac{998}{10} = 99.8 \\
    \implies M                           & > 98.8
  \end{align*}
  Therefore \fbox{$M = 99$} suffices.
\end{solution}

\newpage
\question[10]
Determine the radius of convergence and interval of convergence, but don't check the endpoints
\\
$\sum_{n=1}^\infty \frac{2^n}{n} (4x-8)^n$.
\begin{solution}
  \begin{align*}
    \lim_{n\to\infty} \abs{\frac{a_{n+1}}{a_n}}
     & = \lim_{n\to\infty} \abs{\frac{\frac{2^{n+1}}{n+1}(4x-8)^{n+1}}{\frac{2^n}{n}(4x-8)^n}} \\
     & = \lim_{n\to\infty} \abs{2(4x-8) \frac{n}{n+1}}                                         \\
     & = 2\abs{4x-8} < 1                                                                       \\
     & \implies  8 \abs{x-2} < 1
    \implies \abs{x-2} < \frac18
  \end{align*}
  The radius of convergence is \fbox{$R = \frac18$} \\
  The interval of convergence is \fbox{$\left(\frac{15}{8}, \frac{17}{8}\right)$}
\end{solution}

\question[10]
Show whether the series converges absolutely or conditionally or diverges.
Name all tests used.
\\
$\displaystyle \sum_{n=6}^\infty (-1)^n \frac{4+\sqrt n}{n^3-5n^2}$
\begin{solution}
  Using Limit Comparison Test with $\sum \frac{1}{n^{5/2}}$ (which converges by $p$-series test with $p = \frac52 > 1$):
  \[
    \lim_{n\to\infty} \abs{\frac{4+\sqrt{n}}{n^3-5n} \cdot n^{5/2}}
    = \lim_{n\to\infty} \abs{\frac{n^3+4n^{5/2}}{n^3-5n}} = 1
  \]
  By Limit Comparison Test, the given series \fbox{converges absolutely}
\end{solution}

\newpage
\question[10]
Find the Taylor Series for $f(x) = e^{-x}$ about $x = -4$.
\begin{solution}
  \begin{alignat*}{2}
    f(x)   & = e^{-x}  & f(-4)   & = e^4    \\
    f'(x)  & = -e^{-x} & f'(-4)  & = -e^{4} \\
    f''(x) & = e^{-x}  & f''(-4) & = e^{4}
  \end{alignat*}
  The Taylor series is thus
  \[
    e^4 -e^4(x+4) + \frac{e^4}{2!} (x+4)^2 - \frac{e^4}{3!} (x+4)^3 \cdots
    = \boxed{\sum_{n=0}^\infty \frac{(-1)^n e^4}{n!} (x+4)^n}
  \]
\end{solution}

\question[10]
Find the limit of the sequence or state that it diverges.
$\lim\limits_{n\to\infty} n^2e^{-n}$
\begin{solution}
  \begin{align*}
    \lim\limits_{n\to\infty} n^2e^{-n}
    = \lim_{n\to\infty} \frac{n^2}{e^n}
    \overset{L'H}{=}
    \lim_{n\to\infty} \frac{2n}{e^n}
    \overset{L'H}{=}
    \lim_{n\to\infty} \frac{2}{e^n}
    = \boxed{0}
  \end{align*}
\end{solution}

\newpage
Determine whether the following series converge conditionally, converge absolutely, or diverge. Name all tests used.
\question[10]
$\displaystyle \sum_{n=1}^\infty \frac{(-1)^n}{n^{5/3}}$
\begin{solution}
  $\displaystyle \sum \frac{1}{n^{5/3}}$ converges by $p$-series test ($p = \frac53 > 1$), so the given series \fbox{converges absolutely}
\end{solution}

\question[10]
$\displaystyle\sum_{n=1}^\infty \frac{(-1)^n}{\sqrt{n}}$.
\begin{solution}
  The absolute value version of the series
  $\displaystyle \sum \frac{1}{\sqrt n}$ diverges by $p$-series test ($p = \frac12 < 1$).

  Trying the Alternating Series Test:
  \[
    \lim_{n\to\infty} \frac{1}{\sqrt n} = 0
    \qand \frac{1}{\sqrt{n+1}} < \frac{1}{\sqrt n}
  \]
  so $\displaystyle \sum \frac{(-1)^n}{\sqrt n}$ converges by the Alternating Series Test.

  Thus the given series \fbox{converges conditionally}
\end{solution}

\newpage
\question[10]
Find a power series representation for the following function and determine its interval of convergence. \\
$f(x) = \frac{x}{5-x}$
\begin{solution}
One answer is:
\begin{align*}
  \frac{x}{5-x}
  = \frac{x -5 + 5}{5-x}
  = -1 + \frac{5}{5-x}
   & = -1 + \frac{5}{1-(x-4)}                            \\
   & = \boxed{-1 + 5 \sum_{n=0}^\infty (x-4)^n} \qquad (|x-4|<1)
\end{align*}
which has interval of convergence \fbox{$(3,5)$}

Another answer is:
\begin{align*}
  \frac{x}{5-x}
  = x \cdot \frac{1}{5- x}
  = \frac x5 \cdot \frac{1}{1- \frac x5}
   & = \frac x5 \sum_{n=0}^\infty \left(\frac x5\right)^n \qquad \Bigl(\Bigl|\frac x5\Bigr| < 1 \implies \abs{x} < 5 \Bigr) \\
   & = \boxed{\sum_{n=0}^\infty \left(\frac x5\right)^{n+1}} \qquad (\abs{x}<5)
\end{align*}
which has interval of convergence \fbox{$(-5,5)$}
\end{solution}

\question[10]
Determine whether the following series converge conditionally, converge absolutely, or diverge. Name all tests used. \\
$\sum_{n=1}^\infty \left(\frac{3n+1}{4-2n}\right)^n$
\begin{solution}
  Using root test:
  \[
    \lim_{n\to\infty} \sqrt[n]{\abs{\left(\frac{3n+1}{4-2n}\right)^n}}
    = \lim_{n\to\infty} \abs{\frac{3n+1}{4-2n}} = \frac32 > 1
  \]
  Thus by root test, the series \fbox{diverges}
\end{solution}

\newpage
\question[10]
Solve the differential equation for initial conditions $y(1) = 4$ 
\\
(Not required to solve for $y$).
\\
$\dod{y}{x} = x^2 y - 3x^2$
\begin{solution}
  \begin{align*}
    = x^2(y-3) \\ 
    \int \frac{1}{y-3} \dif y &= \int x^2 \dif x \\ 
    \ln|y-3| &= \frac{x^3}{3} + C_1 \\ 
    |y-3| &= e^{\frac{x^3}{3} + C_1} 
    = e^{\frac{x^3}{3}} e^{C_1}
    = C_2 e^{\frac{x^3}{3}} \\
    y-3 &= C_3 e^{\frac{x^3}{3}} \\ 
    y(x) &= C e^{\frac{x^3}{3}} + 3
  \end{align*}
  Plugging in the initial condition:
  \[
    4 = C e^{\frac13} + 3 \implies 1 = C e^{\frac13} \implies C = e^{-\frac13}
  \]
  The solution is thus
  \[
    \boxed{y(x) = e^{-\frac13} e^{\frac{x^3}{3}} + 3}
  \]
\end{solution}

\end{questions}
\end{document}