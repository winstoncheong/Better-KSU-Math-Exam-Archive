\documentclass[12pt,answers]{exam}

\usepackage{amsmath,amsfonts,amssymb,mathtools,physics,commath}
\usepackage{todonotes}
\usepackage{float}
\usepackage{multicol}
\usepackage{polynom}
\usepackage{siunitx}
\usepackage{cancel}

\newcommand{\inv}{^{-1}}

\pagestyle{headandfoot}
\firstpageheadrule
\runningheadrule
\firstpageheader{Math 221}{Exam 3|Solutions, Page \thepage\ of \numpages}{April 14, 2020}
\runningheader{Math 221}{Exam 3|Solutions, Page \thepage\ of \numpages}{April 14, 2020}
\runningfooter{}{}{}

\begin{document}
% \maketitle
\begin{questions}

\question
Evaluate the following.
\begin{parts}
    \part[8]
    $\displaystyle \dod{}{x} x^3 \tanh\inv(e^{2x})$, where $\tanh\inv$ is the inverse tanh function.
    \begin{solution}
        \[
            3x^2 \tanh\inv(e^{2x}) 
            + x^3 \frac{1}{1-(e^{2x})^2} \cdot e^{2x} \cdot 2
        \]
    \end{solution}
    \part[8]
    $\displaystyle \int (2+\cosh x)(1+\sinh x) \dif x$
    \begin{solution}
        \begin{align*}
            \MoveEqLeft \int (2+\cosh x)(1+\sinh x) \dif x \\ 
            &= \int (2 + 2 \sinh x + \cosh x + \sinh x \cosh x) \dif x \\
            &= 2x + 2 \cosh x + \sinh x + \int \sinh x \cosh x \dif x  
            & \left( 
\begin{array}{c}
    u = \sinh x; \\
    \dif u = \cosh x \dif x
\end{array}
                \right)\\ 
            &= 2x + 2 \cosh x + \sinh x + \int u \dif u \\
            &= 2x + 2 \cosh x + \sinh x + \frac{u^2}{2} \\ 
            &= \boxed{2x + 2 \cosh x + \sinh x + \frac{(\sinh x)^2}{2} +C }
        \end{align*}
        \textit{Remark:}
           Alternatively, the $u$-sub can be performed with $u = \cosh x$, $\dif u = \sinh x \dif x$. 
           The final result will have the term $\frac{(\cosh x)^2}{2}$ instead of $\frac{(\sinh x)^2}{2}$. 
           These are off by a constant since $\cosh^2 x - \sinh^2 x = 1$.
    \end{solution}
\end{parts}

\newpage
\question
Consider the differential equation
\[
    \dod{y}{x} = \frac{xy}{\ln y}
\]
\begin{parts}
	\part[8]
    Find the general solution. Solve for $y$ explicitly.
	\begin{solution}
        \begin{align*}
            \int \frac{\ln y}{y} \dif y &= \int x \dif x \\ 
            \frac{(\ln y)^2}{2} &= \frac 12 x^2 + C_1 \\ 
            (\ln y)^2 &= x^2 + C_2 \\ 
            \ln y &= \pm \sqrt{x^2 + C_2} \\ 
            \Aboxed{y(x) &= \exp(\pm \sqrt{x^2+C})}
        \end{align*}
	\end{solution}
	\part[2]
    Find the solution satisfying the initial condition $y(0) = e$ where $e$ is the natural log base.
	\begin{solution}
        \begin{align*}
            % y(x) &= \exp(\pm \sqrt{x^2+C}) \\
            y(0) &= \exp(\pm \sqrt{0+C}) = e 
            \implies \pm \sqrt{C} = 1 
            \implies C = 1 \\
            \Aboxed{y(x) &= \exp(\sqrt{x^2+1})}
        \end{align*}
	\end{solution}
\end{parts}

% \newpage
\question[6]
Find the limit of the sequence or state that it diverges.
\begin{parts}
    \part[6] 
    $\displaystyle \lim_{n\to\infty} e^{-2n}(n^2+1)$
    \begin{solution}
        \begin{align*}
            \lim_{n\to\infty} \frac{n^2+1}{e^{2n}}
            \overset{LH}{=} 
            \lim_{n\to\infty} \frac{2n}{2e^{2n}}
            \overset{LH}{=}
            \lim_{n\to\infty} \frac{2}{4e^{2n}}
            = \boxed{0}
        \end{align*}
    \end{solution}
    \part[6]
    $\displaystyle \lim_{n\to\infty} \frac{\sin(5/n)}{\sin(3/n)}$
    \begin{solution}
        \begin{align*}
            \lim_{n\to\infty} \frac{\sin(5/n)}{\sin(3/n)}
            \overset{LH}{=}
            \lim_{n\to\infty} \frac{\cos(5/n)\cdot -5n^{-2}}{\cos(3/n) \cdot -3n^{-2}}
            =
            \lim_{n\to\infty} \frac{5\cos(5/n)}{3\cos(3/n)}
            = \boxed{\frac{5}{3}}
        \end{align*}
    \end{solution}
\end{parts}

\newpage
\question
Evaluate the series.
\begin{parts}
    \part[6]
    $\displaystyle \sum_{n=0}^\infty \frac52 \cdot \left(\frac25\right)^n = \frac52 + 1 + \frac25 + \frac{4}{25} + \frac{8}{125} + \cdots $
    \begin{solution}
        Geometric series with first term $a = \frac 52$ and ratio $r = \frac25$.
        Sum is
        \[
            \frac{a}{1-r}
            = \frac{\frac 52}{1-\frac25}
            = \frac 52 \cdot \frac 53
            = \boxed{\frac{25}{6}}
        \]
    \end{solution}
    \part[8]
    $\displaystyle \sum_{n=2}^\infty \frac{2}{n^2-n}$
    \begin{solution}
        \begin{align*}
            \sum_{n=2}^\infty \frac{2}{n^2-n}
            &= \sum_{n=2}^\infty \left( \frac{2}{n-1} - \frac 2n \right) \\ 
            &= \left( \frac21 - \cancel{\frac22} \right) + \left( \cancel{\frac 22} - \cancel{\frac 23} \right) + \left( \cancel{\frac 23} - \frac 24\right) + \cdots \\ 
            &= \boxed{2}
        \end{align*}
    \end{solution}
\end{parts}

% \newpage
\question[7]
Use the limit comparision test to determine whether the following series converges or diverges. Show all work to justify your answer.
$\displaystyle \sum_{n=1}^{\infty} \frac{n^2-n}{3n^{5/2} + 217}$
\begin{solution}
    Compare against $\displaystyle \sum_{n=1}^\infty \frac{1}{n^{1/2}}$
    \[
        \lim_{n\to\infty} \frac{\dfrac{n^2-n}{3n^{5/2} + 217}}{\dfrac{1}{n^{1/2}}}
        = 
        \lim_{n\to\infty} \frac{n^{5/2}-n^{3/2}}{3n^{5/2} + 217} = \frac 13
    \]
    and $\displaystyle \sum_{n=1}^\infty \frac{1}{n^{1/2}}$ diverges by $p$-series test. 
    So by the Limit Comparison Test, \\
$\displaystyle \sum_{n=1}^{\infty} \frac{n^2-n}{3n^{5/2} + 217}$ \fbox{diverges}
\end{solution}

\newpage
\question
Determine whether the following series converge or diverge. Show all work to justify your answers.
\begin{parts}
    \part[7]
    $\displaystyle \sum_{n=1}^\infty \frac{1}{\sqrt n}$
    \begin{solution}
        \fbox{Diverges} by the $p$-series test ($p = \frac12 < 1$)
    \end{solution}
    \part[7]
    $\displaystyle \sum_{n=2}^\infty \frac{1}{n(\ln(n))^2}$
    \begin{solution}
        \fbox{Converges} by the integral test:
        $f(x) = \dfrac{1}{x(\ln x)^2}$ is positive, continuous, and decreasing for $x \ge 2$, and
        \[
            \int_2^\infty \frac{1}{x (\ln x)^2} \dif x 
            = \int u^{-2} \dif u 
            = -u^{-1} 
            = \eval{-\frac{1}{\ln x}}_2^\infty 
            = - 0 + \frac{1}{\ln 2} \qquad \text{converges}
        \]
    \end{solution}
    \part[7]
    $\displaystyle \sum_{n=1}^\infty (-1)^{n+1} \cos(n\pi/2)$
    \begin{solution}
        \fbox{Diverges} by divergence test, since
        $\displaystyle \lim_{n\to\infty} (-1)^{n+1} \cos(n\pi/2)$ does not exist (the sequence cycles between the values 0, 1, 0, and $-1$)
    \end{solution}
\end{parts}

% \newpage
\question[6]
The infinite series $\displaystyle S = \sum_{n=1}^\infty \frac{(-1)^{n+1}}{3n-1}$ is estimated using the $M$-th partial sum $S_M$. Find the minimum $M$ that guarantees that $|S - S_M| < 0.01$.
\begin{solution}
    This is an alternating series $\displaystyle \sum_{n=1}^\infty (-1)^{n+1} a_n$, with $a_n = \dfrac{1}{3n-1}$.
    The estimate for error is 
    $|S - S_M| < a_{M+1}$. We want
    \begin{align*}
        \frac{1}{3(M+1)-1} < 0.01 = \frac{1}{100}
        &\implies 100 < 3(M+1) - 1 \\  
        &\implies \frac{101}{3} - 1 < M \\ 
        &\implies M > \frac{98}{3} \approx 32.66
    \end{align*}
    so $\boxed{M = 33}$ works.
\end{solution}

\question
Determine whether the following series converge conditionally, converge absolutely, or diverge. Justify your answer.
\begin{parts}
    \part[7]
    $\displaystyle \sum_{n=1}^\infty \frac{(-1)^n}{n}$
    \begin{solution}
        $\displaystyle \sum_{n=1}^\infty \frac 1n$ diverges (harmonic series).
        \\
        $\displaystyle \lim_{n\to\infty} \frac{1}{n} = 0$ and $\dfrac{1}{n+1} \le \dfrac1n$ for $n \ge 1$, so by the Alternating Series Test,
        $\displaystyle \sum_{n=1}^\infty \frac{(-1)^n}{n}$ converges.
        \\
        Thus
        $\displaystyle \sum_{n=1}^\infty \frac{(-1)^n}{n}$
        \fbox{converges conditionally}
    \end{solution}
    \part[7]
    $\displaystyle \sum_{n=1}^\infty \frac{\sin(2n)}{n^2+n}$
    \begin{solution}
        \[
    \sum_{n=1}^\infty \left|\frac{\sin(2n)}{n^2+n} \right|
    \le 
    \sum_{n=1}^\infty \frac{1}{n^2+n} 
    \le 
    \sum_{n=1}^\infty \frac{1}{n^2} 
        \]
        The rightmost series converges by $p$-series test ($p = 2 > 1$). 
        \\
        Hence by Direct Comparison Test, 
    $\displaystyle \sum_{n=1}^\infty \left|\frac{\sin(2n)}{n^2+n} \right|$ converges.
    \\
    Thus by definition, 
    $\displaystyle \sum_{n=1}^\infty \frac{\sin(2n)}{n^2+n}$ \fbox{converges absolutely}
    \end{solution}
\end{parts}
\end{questions}
\end{document}