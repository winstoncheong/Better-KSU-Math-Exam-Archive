%From zou@math.ksu.edu Fri May  1 16:46:28 1998
%Date: Fri, 1 May 1998 15:45:57 -0500 (CDT)
%From: Qisu Zou <zou@math.ksu.edu>
%To: Math Account <math@math.ksu.edu>
%Subject: Re: Num Analy Qual - Fall 1996


%********************************************************************
% na96f.tex
\def\hcorrection#1{\advance\hoffset by #1 }
\def\vcorrection#1{\advance\voffset by #1 }
 
%\documentstyle{article}
\documentclass{article}
 
\vcorrection{-1.0in}
\hcorrection{-.9in}
\textwidth 7.0in
\textheight 9.0in
 
\begin{document}
 
\begin{large}
 
\begin{center}
    \begin{Large}
        {\bf NUMERICAL ANALYSIS QUALIFYING EXAM \\
            Fall, 1996}\\
    \end{Large}
\end{center}
 
\vspace{.1in}
 
%Test questions in description mode
 
(do at lest 3 problems from problems 1-4, and do at least
3 problems from problems 5-8, you may do as many as you can)  \\


 \begin{enumerate}

\item Prove the following theorem:  If $f \in C^2(a,b) $,  
$f^{'}(x) f^{''}(x) \neq 0$, and $f(x)$ has a zero  in $(a,b)$,
then the zero is unique in $(a,b)$, and the Newton iteration 
will converge to
it if the starting value $x_0$ and the first approximation $x_1$
are both in $(a,b)$.  
(You may just do a special  case where $f^{'}(x) < 0,  f^{''}(x) < 0$
in $(a,b)$)


\item Suppose a numerical integration formula $I_n$ using $n$ 
subintervals to approximate the definite integral 
$I = \int_a^b f(x) dx$  has an error given by 
$I-I_n \doteq \frac{c}{n^p} $ where $c, p$ are  constants.
Derive the computable estimate
\[   \frac{I_{2n} - I_n}{I_{4n} - I_{2n}} \doteq  2^p  \]
This gives a practical means of checking the value of $p$,
using three successive values $I_n, I_{2n}, $ and $I_{4n}$.

\item  By {\bf considering the proof}  of 
\[  f(x) - p_n(x) = \frac{f^{(n+1)}(\xi)}{(n+1)\!} (x-x_0)(x-x_1) \ldots
 (x-x_n) ,     \]
where $p_n(x)$ is the polynomial of degree less than or
equal to $n$,which interpolates $f(x)$ at $n+1$ nodes
$x_0, x_1, \ldots, x_n$.
Find that the error formula for
\[ f(x) - p_m(x) , \]
where $p_m(x)$ is a polynomial of degree greater than $n$,
which interpolates $f(x)$ at $n+1$ nodes
$x_0, x_1, \ldots, x_n$.
\item 

Prove the following theorem:
Define a  set of functions 
$$P_M^n  \equiv \{p \in P^n | \max_{x \in [a,b]} |p(x)| \leq M \}
$$ where $P^n$ is the linear space of the polynomials of degree less than or 
equal to $n$.
Then there is a constant $C > 0$ such that for every $p \in P_M^n$
and $x \in [a, b]$ and any positive integer $k$,
$$ |\frac{d^k p(x)}{dx^k}| \leq C. $$
(Hint: Chebyshev polynomials of degree 0, 1, $\cdots, n$ form
a basis for $P^n$)


\item
 A matrix norm is defined as
\[  \| A\|_{\infty}  = \max_{1\leq i \leq n} \sum_{j=1}^n |a_{ij}| . \]
Prove or disprove: $ \| AB \|_{\infty} = \|A\|_{\infty} \|B\|_{\infty}$.
What about the special case:  $\| A^2 \|_{\infty} =
\|A\|_{\infty} \|A\|_{\infty}$?



 \item Find the explicit form for the iterative matrix in the 
Gauss-Seidel iterative method for solving a linear system
$A {\bf x} = {\bf b}$ when

\[ A =  \left[ \begin{array}{rrrrrrrr}
            2    &-1  &     & & & & &      \\
           -1    &2  &-1                           \\
                 &-1 &2     &-1                     \\
                 &   &-1 &\cdot  &\cdot          \\                       
                 &   &   &\cdot &\cdot  &\cdot       \\                      
                 &   &  &      &\cdot &\cdot  &-1    \\                
                 &   &  &      &   &-1 &2  &-1       \\                      
                 &   &  &      &   &  &-1 &2  
         \end{array}    \right] .              \]

\item Suppose $A$ is an invertible matrix and that $B$ is a 
 matrix with
$ \|B - A^{-1}\| \leq \delta \| A^{-1} \| $. Let $\{ {\bf x}_n 
\}_{n=0}^{\infty} $ be the sequence of vectors generated by the algorithm

\hspace{1in} (i) ${\bf r}_n = {\bf b} - A {\bf x}_n$

\hspace{1in} (ii) ${\bf x}_{n+1}={\bf x}_n + B {\bf r}_n$

with a given starting value ${\bf x}_0$. Give a sufficient condition on 
the size of $\delta$ for 
the sequence to converge to the solution of the linear system
$A{\bf x} = {\bf b} $ for arbitrary starting value 
${\bf x}_0$. Prove that your 
condition is correct.

\item Describe an algorithm that reduces a square real matrix to
a lower Hessenberg matrix without changing its eigenvalues.
(A matrix $A$ is lower Hessenberg if $a_{ij}=0$ provided $j-i > 1$.)




 \end{enumerate}

\end{large}

\end{document}


-- 

