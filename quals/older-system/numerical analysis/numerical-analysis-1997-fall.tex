% latex file
\def\hcorrection#1{\advance\hoffset by #1 }
\def\vcorrection#1{\advance\voffset by #1 }

\documentclass{article}
\usepackage{my,amsxtra,amssymb,amsthm}

\vcorrection{-1.0in}
\hcorrection{-0.8in}
\textwidth 6.0in
\textheight 9.0in
\begin{document}
%\begin{Large}






\begin{center}\begin{LARGE}
{\bf Numerical Analysis Qualifying Exam}\\ 
{\bf Fall 1997}\\ \end{LARGE}
\end{center}
\vspace{0.1in}
\noindent\hrulefill\\
\begin{description}

\item[1.]
Explain the difficulty in evaluating $f(x) = x^{-1}$ $(1-\cos x)$
when the absolute value of $x$ is small and find a method without using
the Taylor series to overcome the difficulty.

\item[2.]
Prove the following theorem: If $f \in C^2(a,b)$, $f^\prime (x)$
$f^{\prime \prime} (x) \neq 0$, and $f(x)$ has a zero in $(a,b)$, then the
zero is unique in $(a,b)$, and the Newton iteration will converge to it if
the starting value $x_0$ and the first approximation $x_1$ are both
in $(a,b)$. (You may just do a special case where
$f^\prime (x) < 0$, $f^{\prime \prime} (x) <0$ in $(a,b)$)

\item[3.]
Prove the following theorem for Gaussian quadrature:

Let $I(f) = \int^b_a f(x) w(x) dx$, where $w(x)$ is a positive weight
function, be approximated by a quadrature formula
$I_n(f) = \sum^n_{i=1} A_i f(x_i)$, where $x_i \in [a,b]$, $i=1:n$
are distinct. Then the quadrature formula $I_n(f)$ has a maximum degree
of precision of $2n-1$. This is attained if and only if
$x_i, x_2, \dots, x_n$ are the zeros of $p_n(x)$, the $n\hbox{th}$ orthogonal
polynomial, with the inner product
$$ (f,g) = \int^b_a f(x) g(x) w(x) dx.$$
(Hint: if $f(x)$ is a polynomial of degree $m$ with $m \geq n$, then
$f[x_1, x_2, \dots, x_n, x]$ is a polynomial of degree $m-n$)

\item[4.]
Let $f(x)$ be a piecewise constant function on $[a,b]$:
$$f(x) = \begin{cases}
          \alpha &a \leq x \leq c \\
          \beta  &c < x \leq b
          \end{cases}$$
where $a<c<b$ and $\alpha \neq \beta$. Let $x_0 < x_1 < \dots < x_n$ be
$n+1$ points and $j$ be an integer with $0 \leq j \leq n-1$ such that
$a < x_0 < \dots < x_j < c < x_{j+1} < \dots < x_n < b$
and that $P_n(x)$ is the polynomial of degree less than or equal to $n$
interpolating $f$ at the $n+1$ points $x_i$, $i=0, \dots , n$.
Show that $P_n(x)$ is monotone in the interval $[x_j, x_{j+1}]$.
(Hint: Count the number of zeros of $\frac{dP_n(x)}{dx}$.)

\item[5.]
Suppose $A \in R^{n \times n}, b \in R^n$, and $P A = LU$, where $P$ is a
permutation matrix, $L$ is lower triangular, and $U$ is upper triangular.
$A,b,P,L,U$ are known. Determine the purpose of the following algorithm:

For $j=1:k$

\quad solve for $y$ in $Ly = Pb$ \quad overwrite $b$ with $y$

\quad solve for $x$ in $Ux = b$ \quad overwrite $b$ with $x$

end

(Hint: consider $k=1$ first)

\item[6.]
Let $A$ be an invertible $n \times n$ matrix, and let $u$ and $v$ be two
vectors in $R^n$. Find the necessary and sufficient conditions on $u$
and $v$ in order that the matrix
$$B= \left[ \begin{array}{cc} A&u \\ v^t&0 \end{array} \right]_.$$
be invertible, and give the inverse in terms of $A, u,v$ when it exists.
(Hint: multiply $B$ by a suitable matrix to have a simpler matrix to handle)

\item[7.]
Suppose that an $n$ by $n$ nonsingular matrix $A$ has distinct eigenvalues
$\lambda_1, \dots, \lambda_n$ with corresponding right eigenvectors
$u_1, \dots, u_n$ and left eigenvectors $v_1, \dots, v_n$. Suppose
$$c_i = \parallel u_i \parallel_2 \parallel v_i \parallel _2 / |v_i^T u_i|,
  i=1, \dots, n.$$
Show that the solution of $Ax=b$ satisfies
$$\parallel x \parallel_2 \leq \parallel b \parallel_2 \sum^n_{i=1}
  \frac{c_i}{|\lambda_i|}.$$
(Hint: write $x$ in terms of $u_i$)

\item[8.]State Schur's theorem and use it to show: Let $A$ be an $n \times n$
complex matrix, it can be similar to a upper triangular matrix with $|$off-diag.
entries$|$ $\leq \varepsilon$, (Hint: on the result of Schur's, construct
suitable diagonal matrices)







\end{description}    
%\end{Large}
\end{document}














