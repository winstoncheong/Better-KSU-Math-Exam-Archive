%From hyang@math.ksu.edu Mon Feb 23 16:27:50 1998

% the new version

% na98s.tex
\def\hcorrection#1{\advance\hoffset by #1 }
\def\vcorrection#1{\advance\voffset by #1 }
 
\documentstyle{article}
 
\vcorrection{-1.0in}
\hcorrection{-.9in}
\textwidth 7.0in
\textheight 9.0in
 
\begin{document}
 
\begin{large}
 
\begin{center}
    \begin{Large}
        {\bf NUMERICAL ANALYSIS QUALIFYING EXAM \\
            Spring, 1998}\\
            Yang \& Zou \\
    \end{Large}
\end{center}
 
\vspace{.1in}
 
%Test questions in description mode
 

(do at lest 3 problems from problems 1-4, and do at least
3 problems from problems 5-8, you may do as many as you can)  \\


 \begin{enumerate}

\item  
   Show that
\begin{enumerate}
     \item $1 + x \leq e^x, \ \forall \ {\mbox real } \  x .  $
     \item $ e^x \leq 1 + 1.01 x, \ \  \forall \  0 \leq x \leq 0.01. $
\end{enumerate}
   And use the results to show
\[ (1+u)^n \leq 1 + 1.01 nu     \;\;\;\;\; \mbox{if}\;\; 0 \leq nu \leq 0.01 .
\;\;\;\;\;\;      \]


 \item By constructing a fixed-point iteration,
find the value of  $x$ given by
\[   x = \sqrt{p+\sqrt{p+\sqrt{p + \cdots }}}   \]
where $p$ is a positive number. Prove the convergence of the
fixed-point iteration. 

  \item
 Construct a near-minimax polynomial of degree $\leq$ 2 for the
function $g(t) = e^t$ on the interval $t \in [0,1]$ and estimate its maximum
error. You can express the result in terms of the expenential function. 
(hint: $\cos (n+1)\theta = \cos \theta \cos n\theta  -\cos (n-1)\theta $)




 \item
Given the trapezoidal rule and Simpson's rule as:
\[
\int_{x_0}^{x_0+h} f(x) dx = \frac{h}{2} [f(x_0) + f(x_0+h)] - \frac{h^3}{12
}
f^{(2)}(\xi) ,
\]
\[
\int_{x_0}^{x_0+2h} f(x) dx = \frac{h}{3} [f(x_0) + 4f(x_0+h)+f(x_0+2h)] -
\frac{h^5}{90} f^{(4)}(\xi) ,
\]
\begin{description}
\item[(1)]
Describe the composite trapezoidal rule and composite Simpson's rule for
evaluating integrals $\int_a^b f(x) dx$ using $n$ subintervals.
\item[(2)]
 Derive an estimate for the error in the composite trapezoidal rule in terms
of the length of the subintervals into which $[a,b]$ is divided.
\par
\item[(3)] Derive the asymptotic error formula for the composite
Simpson's rule
\[  E_n(f) \doteq -\frac{h^4}{180} [f^{(3)}(b)-f^{(3)}(a)], \] 
where $h = (b-a)/n$.
\par
\end{description}


 \item  Let $U$ and  $V$ be two $3 \times 3$ matrices such that
\[ UV  = [w_{ij}] =  \left[ \begin{array}{rrr}
            w_{11}    &w_{12}  &0     \\
            w_{21}    &w_{22}  &0     \\
            0         &0       &0     \\
         \end{array}    \right] ,  \ \ \mbox{where} \ \
 w_{11} w_{22} \neq  w_{12} w_{21} .            \]
Show that either the last row of $U$ is the zero row vector
or the last column of $V$ is the zero column vector.
(Hint: If $U$ is singular then there exists a nonzero 3-tuple vector
$p$ such that $p^T U = 0$)


\item  Let $A$ be an $n \times n$ symmetric real positive definite
matrix. Show that there exist $2^n$ real lower-triangular matrices $L$
such that $A = LL^T$?

  \item
Assume that ${\bf w} \in {\bf R}^n$, and that $ \| {\bf w} \|_2=1$.
What are the eigenvalues, eigenvectors, and determinant of a Householder
matrix \\ $I - 2{\bf ww}^T$? 



 \item
By using the singular value decomposition, show that any square real
matrix $A$ can be 
written as
$A=QS$ where $Q$ is an orthogonal matrix and $S$ is a semipositive
definite matrix.


 \end{enumerate}




\end{large}

\end{document}

%
% \item
%Suppose $ \lambda_1 > \lambda_2 > \cdots > \lambda_n$, and ${\bf v} \in
%\Re^n$ such that its components $v_i$ are nonzero for all $i$.
%Let $k$ be an integer with $ 1 \leq k \leq n$. Fix $v_i$ for $i \neq k$
%and allow $v_k$ to varry. Show that the matrix $A({\bf v}) = 
%\mbox{diag}(\lambda_1, \lambda_2, \ldots, \lambda^n) + {\bf v}{\bf v}^T$ has 
%an eigenvalue $\lambda({\bf v})$ that satisfies
%$$\lambda({\bf v}) = \lambda_k + C({\bf v}) v_k^2$$
%where 
%$$\lim_{v_k \rightarrow 0} C({\bf v}) = \frac{1}{1-\sum_{i=1, i \neq k}
%\frac{v_i^2}{\lambda_k - \lambda_i}}.$$
%
