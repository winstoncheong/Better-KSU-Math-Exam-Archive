\def\hcorrection#1{\advance\hoffset by #1 }
\def\vcorrection#1{\advance\voffset by #1 }


%\input prepictex.tex
%\input pictex.tex
%\input postpictex.tex

%\documentstyle[bbb]{report}
\documentclass[bbb]{report}
\usepackage{amsxtra,amssymb,amsthm,amsmath,latexsym}

\vcorrection{-.5in}
\hcorrection{-1in}

\topmargin  0in
%\textwidth 7in
%\textheight 10truein
\textheight 9truein
\textheight 9truein
\textwidth 6.5in
\mathsurround=2pt



\pagestyle{empty}

\def\ds{\displaystyle}

\def\R{{\Bbb R}}
\def\Z{{\Bbb Z}}
\def\C{{\Bbb C}}
\def\D{{\Bbb D}}
\def\N{{\Bbb N}}
\def\F{{\cal F}}
\def\M{{\cal M}}

\begin{document}


\begin{large}

\vspace{.12in}

\begin{center}
  REAL ANALYSIS QUALIFYING EXAM \\
  Fall 2001 \\
\end{center}


\vspace{.1in}


\begin{large}
{\bf Answer as many as possible.
Throughout, $(X, {\cal M} ,\mu)$ denotes a measure space,
$\mu$ denotes a positive measure unless otherwise specified, and all
functions are assumed to be measurable.}
\end{large}

\vspace{.1in}

\renewcommand\baselinestretch{1.25}

\begin{description}

\item[1.] \, \, (a)
Does $\int^\infty_0 \frac{\sin x}{x} dx$ exist as an (improper) Riemann
integral? Prove your answer.

\item[\quad] (b)
Does $\int^\infty_0 \frac{\sin x}{x} dx$ exist as a Lebesgue integral?
Prove your answer.

\vspace{.05in}

\item[2.]
Suppose $\lambda$ denotes Lebesgue measure on $\Bbb R$ and $A$ is a Borel
set with
$\lambda (A) > 0$. Show that for every $0<r<1$, there is a bounded open
interval $I$
with $\lambda (A\cap I) > r \lambda (I)$.

\vspace{.05in}

\item[3.]
Let $\mu$ be a complex measure on ${\cal M}$. Show that there exists a set
$A \subset {\cal M}$ such that

\item[\quad] (i)
$B \subset A$ implies $\text{Re} (\mu (B)) \geq \text{Im} (\mu (B))$

\item[\quad] (ii)
$B \subset A^c$ implies $\text{Im} (\mu (B)) \geq \text{Re} (\mu (B))$.

\vspace{.05in}

\item[4.]
Prove that $(L^\infty([0,1]))^* \neq L^1([0,1]).$
Hint: One way to prove this is to show there is a bounded linear
functional $\Lambda\neq 0$
such that $\Lambda |_{C([0,1])} \equiv 0$. (Here, $C([0,1])$ denotes
continuous functions on the interval $[0,1].$)  \ Of course, you should
explain why the existence of this functional proves that
$(L^\infty([0,1]))^* \neq L^1([0,1])$.

%\item[4.]
% Prove the Riemann--Lebesgue lemma: If $f \in L^1([-\pi, \pi])$ and if
% $\hat f(n)=\frac{1}{2\pi}\int_-\pi^\pi f(x) e^{inx}dx$, then
% $\hat f(n) \to 0$ as $|n| \to \infty$.
\vspace{.05in}

\item[5.] Suppose 
$(X,{\cal M}, \mu)$ is a $\sigma$-finite measure space, $\mu$ positive,
and $f:X \to [0,
\infty)$
an ${\cal M}$
measurable function. Suppose $G: [0, \infty) \to [0, \infty)$ is 
increasing and  absolutely continuous, $G(0)=0$.

Prove that
$$\int_X G(f(x)) d \mu = \int^\infty_0 G' (t) \, \mu (\{x : f(x) >
t\}) dt.$$

\vspace{.05in}

\item[6.]
Suppose $\nu, \mu$ are positive measures with $\nu$ finite.  Show that the
following two
statements are equivalent.

\item[\quad] (a)
$\nu << \mu$

\item[\quad] (b)
For every $\varepsilon > 0$, there exists a $\delta>0$ that $\nu(B) <
\varepsilon$ whenever
$B \in {\cal M}$ and $\mu (B) < \delta$.

\vspace{.05in}

\item[7.]
Suppose $\{f_n\}$ is a  sequence of nonnegative measurable functions on
$X$
such that \linebreak
$\lim_{n \to \infty} f_ n(x) = f(x)$ a.e. and
$\lim_{\substack {n \to \infty}} \int_X f_n d \mu = \int_X fd \mu <
\infty$.

Prove that $\lim_{\substack {n \to \infty}} \int_E f_n  d \mu = \int_E f d
\mu$
for every measurable set $E \subseteq X$.

\vspace{.05in}

\item[8.]
Suppose $f$ is a complex measurable function on $X$, $\mu$ is a positive
measure on $X$ and $\varphi (p) = \int_X |f|^p d \mu$,
$0 < p < \infty$. Let $E = \{p : \varphi (p) < \infty \}$.

Show that if $r < p < s,\,  r \in E, \, s \in E$ then $p \in E.$  Show
that the function  $\log \varphi$ is convex on the interior of $E$.

\vfill

\end{description}

\end{large}

\end{document}
