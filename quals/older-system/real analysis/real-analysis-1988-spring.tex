% latex file
\def\hcorrection#1{\advance\hoffset by #1 }
\def\vcorrection#1{\advance\voffset by #1 }

\documentclass{article}
\usepackage{my,amsxtra,amssymb,amsthm}

\vcorrection{-1.0in}
\hcorrection{-0.8in}
\textwidth 6.0in
\textheight 9.0in

\def\R{\mathbb R}
\def\C{\mathbb C}
\def\N{\mathbb N}
\def\Z{\mathbb Z}
\def\Q{\mathbb Q}
\def\T{\mathbb T}
\def\Z{\mathbb Z}

\begin{document}
%\begin{Large}






\begin{center}\begin{LARGE}
{\bf Real Analysis Qualifying Exam}\\ 
{\bf Spring 1989}\\ \end{LARGE}
\end{center}
\vspace{0.1in}
\noindent\hrulefill\\

In the following, $(X, {\cal A}, \mu)$ is a measure space and if $X$ is
a topological space, then ${\cal B}(X)$ is the $\sigma$-algebra of all
Borel subsets of $X$.

\begin{description}
\item[1.] (a)
What does it mean to say that $f:X \to \C$ is ${\cal A}$-measurable?

\item[\quad] (b)
Prove that if $f:g : X \to \C$ are both ${\cal A}$-measurable, then so is
$f+g$.

\item[2.]
Suppose $\{A_n\}^\infty_{n=1} \subset {\cal A}$ and
$\sum^\infty_{n=1} \mu(A_n) < \infty$. Define
$$B=\{x \in X: \{n \in N : x \in A_n\} \hbox{\ is\ } \hbox{infinite}\}.$$
Prove that $B\in {\cal A}$ and $\mu(B) = 0$.

\item[3.]
Suppose $\mu(X) < \infty$ and $\phi : X \to \R$ is ${\cal A}$-measurable.
Define $\nu$ on ${\cal B}(\R)$ by $\nu
(B) = \mu(\phi^{-1} (B))$. Prove that
$\nu$ is a finite measure and that if $f:\R \to \C$ is a bounded
Borel measurable function, then $f \circ \phi$ is ${\cal A}$-measurable and
$$\int_X f \circ \phi d\mu = \int_\R fd\nu.$$

\item[4.] (a)
Let ${\cal S}$ be the smallest $\sigma$-algebra of subsets of $\R^2$ that
contains the family ${\cal F} = \{ A \times B : A$ and $B$ are bounded open
intervals of $\R$\}. Prove that ${\cal S} = {\cal B} (\R^2)$.

\item[\quad] (b)
For $E \subset \R^2$ and $y \in \R$ put
$$E^y = \{x \in \R: (x,y) \in E\}.$$
Prove that if $E \in {\cal B} (\R^2)$, then $E^y \in {\cal B} (\R)$ for all
$y \in \R$.

\item[5.]
Show that if $B$ is a Borel subset of $\R^2$ and almost every vertical
cross-section of $B$ has (one-dimensional) Lebesgue measure 0, then almost
every horizontal cross-section of $B$ has Lebesgue measure 0.

\item[6.]
Prove that if $p \in \N$ and $\mu : {\cal B}(\R^p) \to [0,\infty]$ is a
measure such that $\mu (K) < \infty$ for every compact $K \subset \R^p$, then
$\mu$ is regular. [HINT: Use a Riesz Representation Theorem to find a
``regular relative" of $\mu$.]

\item[7.]
Let $p \in \N$ and let $(\mu_n)^\infty_{n=0}$ be a sequence Borel probability
measures on $\R^p$. Suppose that
$$\lim_{n\to \infty} \int gd\mu_n = \int g d \mu_0$$
whenever $g: \R^p \to \C$ is continuous with compact support. Prove the
following:

\item[\quad] (a)
If $\varepsilon > 0$, then there exists a compact $K \subset \R^p$ such that
$\mu_n(K^\prime) < \varepsilon$ for all $n \geqq 0$. Here
$K^\prime = \R^p \ K$.

\item[\quad] (b)
If $f: \R^p \to \C$ is bounded and continuous, then
$$\lim_{n \to \infty} \int fd\mu_n = \int f d\mu_0.$$

\item[\quad] (c)
If $B \subset \R^p$ is a Borel set whose closure $B^-$ and interior $B^0$
satisfy $\mu_0(B^-)=\mu_0 (B^0)$, then
$$\lim_{n\to \infty} \mu_n (B) = \mu_0 (B).$$

\item[8.]
For $\mu \in M(\R)$, define $\widehat \mu$ on $\R$ by
$$\widehat \mu (t) = \int e^{-itx} d\mu (x).$$
Prove that if $\int |x| d |\mu| (x) < \infty$, then $\widehat \mu$ is
differentiable on $\R$.

\item[9.]
For $\mu \in M(\T)$, define $\widehat \mu$ on $\Z$ by
$$\widehat \mu (n) = \int e^{-inx} d\mu (x).$$
Prove that if $\lim_{n \to +\infty} \widehat \mu (n) = 0$, then
$\lim_{n \to -\infty} \widehat \mu (n) = 0$. [HINT: Don't forget
that $\mu$ is a {\it complex} measure. Remember complex conjugation.
Approximate a Radon-Nikodym derivative by trigonometric polynomials.]

\item[10.]
Let ${\cal A}_0$ be an {\it algebra} of subsets of $X$ (i.e.,
${\cal A}_0$ is closed under finite unions, relative complements and contains
$\emptyset$ and $X$), ${\cal A}$ the smallest $\sigma$-algebra which contains
${\cal A}_0$, and $\mu$ a finite positive measure on ${\cal A}$. Show that
${\cal A}_0$ is $\mu$-dense in ${\cal A}$ in this sense: for $A \in {\cal A}$
and $\varepsilon > 0$ $\exists A_0 \in {\cal A}_0$ such that
$\mu(A \Delta A_0) < \varepsilon$. Here $A \Delta A_0$ means
$(A \backslash A_0) \cup (A_0 \backslash A)$. [HINTS: Look at the class of
all $A$ in ${\cal A}$ which are so approximable. Matters may be facilitated
by translating everything into statements about characteristic functions and
the $L_1$-metric.]







\end{description}    
%\end{Large}
\end{document}














