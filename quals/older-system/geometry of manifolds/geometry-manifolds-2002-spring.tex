%Auckly & Miller, Spring 2002, QE-GM
\documentclass[12pt]{article}
\usepackage{amsxtra,latexsym,amssymb,amsthm}


\setlength{\parindent}{0cm}

\def\ds{\displaystyle}
\def\R{{\mathbb R}}
\def\Z{{\mathbb Z}}
\def\C{{\mathbb C}}
\def\grad{{\hbox{\,grad\,}}}
\def\span{{\hbox{\,span\,}}}
\def\id{{\hbox{\,id\,}}}

%\pagestyle{empty}

\textwidth=6.5truein
\textheight=9.5truein
\hoffset=-1truein
\voffset=-1truein

\begin{document}

\def\R{{\mathbb R}}

\begin{large}
\begin{center}
{\bf
  GEOMETRY OF MANIFOLDS QUALIFYING EXAM}\\
  Spring 2002\\
  (Auckly \& Miller)
\end{center}

\vspace{-.5in}

{
\begin{list}{}
{\itemsep=0em \parsep=0em }\item[]
\end{list}
}

\begin{description}
%\setlength{\itemsep}{0em}\newlength{\parsep}{20em}

\item 1.
Let $\alpha=z^3\,dx\wedge dy- y\,dx\wedge dz\in \Gamma(\wedge^1 \R^3)$.
Let $X=\partial_x+x\partial_z \in \Gamma(T\R^3)$\\
Compute:

\vspace{-.2in}
\begin{tabular}{ll}
\rule{3in}{0in} & \\
a) $d\alpha$        & e) $L_X dz$\\
b) $i_X\alpha$ & f) $L_X\alpha$\\
c) $L_Xdx$     & g) $\ds\int_{S^2}\alpha$ \\
d) $L_X dy$    & [Here $S^2$ is oriented with \\
               &
                  $i_{(x\partial_x+y\partial_y
                   +z\partial_z)}(dx\wedge dy\wedge dz)$.]
\end{tabular}

\item 2.
Find $\int_\Sigma dy\wedge dx + dz\wedge dy + dx\wedge dz$
when
$$\Sigma=\{(x,y,z)\in\R^3 | z=1-(x^2+y^2)^{2002}, \ z\geq 0\}$$
and $\Omega_\Sigma |_{(0,0,1)} = dx\wedge dy$.

\item 3.
Let $X=\R P^2 \vee S^1$ \quad ($\vee$ is the 1 point union.)

\item\quad a)
Compute $\pi_1(X)$.

\item\quad b)
Construct a $2\hbox{-fold}$ cover of $X$, say $\widehat X$,
with $H_2(\widehat X;,\Z)\not= 0$.

\item\quad c)
Compute $H_\ast(X;\Z)$.

\item \quad d)
Compute $H_\ast(\widehat X;\Z)$.

\item 4.
Let $f:\R^2\to\R$; $f(x,y)=x^2-y^2$. Let $g=dx^2+dy^2$.

\item\quad a)
Compute $\grad f$.

\item\quad b)
Let $\alpha_n, \beta_n, \gamma_n:\R\to\R^2$ be integral curves of $\grad f$
with $\alpha_n(0)=(\frac{1}{n^2},1)$, $\beta_n(0)=(\frac{1}{n},\frac{1}{n})$,
$\gamma_n(0)=(1,\frac{1}{n^2})$.
Find expressions for $\alpha_n$, $\beta_n$ and $\gamma_n$.

\item\quad c)
Prove that $\alpha_n(\R)=\beta_n(\R)=\gamma_n(\R)$.

\item\quad d)
Compute $\lim_{n\to\infty} \alpha_n(t)$, $\lim_{n\to\infty}\beta_n(t)$
and $\lim_{n\to\infty}\gamma_n(t)$.

\item 5.
Let $0<a<b$. The equations
$$\begin{aligned}
x&=(b+a\cos \psi)\cos \theta\\
y&=(b+a\cos \psi)\sin \theta\\
z&=a\sin \psi, \quad \theta,\psi\in[0,2\pi]
\end{aligned}$$
describe a surface in $\R^3$.

\item\quad a)
What is this surface?

\item\quad b)
Calculate the Gaussian curvature.

\item\quad c)
Write the equations for geodesics on this surface.

\item 6.
Let $\varphi:M\to N$ be a smooth map between connected, oriented, closed
$n\hbox{-dimensional}$ manifolds. \underbar{Prove that}:
$$ \left( \int_M \varphi^\ast\alpha \right) \left( \int_N\beta \right)
= \left( \int_M\varphi^\ast\beta \right) \left( \int_N\alpha \right)
$$
\underbar{for all} $\alpha$, $\beta \in \Gamma(\wedge^n N)$.

\underbar{Hint}: Think about $H^n(M)$ and $H^n(N)$.

\item 7.
A vector bundle map, $J:TM\to TM$ is called an almost $\C\hbox{-structure}$
if $J^2=-\id$.

\item\quad a)
If a manifold, $M$, admits an almost complex structure,
what can be said about $\dim M$? Why?

\item\quad b)
Prove that any manifold admitting an almost complex structure is
orientable.

\underbar{Hint}: Let $g$ be a Riemannian metric on $M$ and define $w(X,Y)=g(X,J(Y))-g(Y,J(X))$. Use $w$ to construct an orientation.

\item\quad 8.
Let $X=-y\partial_x+x\partial_y+\partial_z$,
$Y=z\partial_x+\partial_y$.
Let $B=\span \{X,Y\}$.

\item\quad a)
Is $B$ integrable? If $B$ is integrable, find the integral manifold through
$(1,0,1)$.

\item\quad b)
Find the flow of $X$.

\item\quad c)
Find the flow of $Y$.


\end{description}
\end{large}
\end{document}
