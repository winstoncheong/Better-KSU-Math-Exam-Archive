%From dbski@math.ksu.edu Fri Aug 21 10:56:58 1998
%QE-F98, Algebra
% latex file
\def\hcorrection#1{\advance\hoffset by #1 }
\def\vcorrection#1{\advance\voffset by #1 }
%use latex209 (older version)

\documentstyle{article}
\pagestyle{plain}


\vcorrection{-1.0in}
\hcorrection{-0.8in}
\textwidth 6.0in
\textheight 9.0in
\begin{document}
\begin{Large}

\input amssym.def   %defines Bbb and gothic (frak)
\input amssym       %defines Bbb and gothic (frak)

\def\Q{{\Bbb Q}}
\def\F{{\Bbb F}}
\def\K{{\Bbb K}}



\begin{center}\begin{LARGE}
{\bf  Algebra Qualifying Exam}\\ 
{\bf August {\bf 25}, 1998}\\ \end{LARGE}
\end{center}
\vspace{0.1in}
\noindent\hrulefill\\
{\bf Instructions:} You are given 10 problems from which you are to do 8.
 Please indicate those  8 problems which you would like  to be graded 
by circling the problem numbers on the  problem sheet. 
{\bf Note:} All rings in this exam are associative and with 1 and 
all integral domains are commutative.
% $\bf Q $ and $ \bf C $ are
%the fields of  rational and complex  numbers respectively.

%$\Bbb Z $ and $ \Bbb Q $ are
%the sets of the integers and rational numbers respectively.

\vspace{0.2in}

\begin{enumerate}

\item Let $G$ be a group of order greater than 2. Show that $G$ has a
non-trivial automorphism.

\item Let $G$ be a group acting transitively on a set $\Omega $. Fix
$\omega\in \Omega$ and let $H=\mbox{Stab}_G(\omega )$. If we let $G$
act on $\Omega\times\Omega $ by $g\cdot (\omega_1,\omega_2) =(g\cdot\omega_1,
g\cdot\omega_2),\ g\in G,\ \omega_1,\omega_2\in\Omega $, 
show that the $G$-orbits on $\Omega\times\Omega $ are in
bijective correspondence with the $H$-orbits on $\Omega $.

\item The group $G$ is called a $CA$-group if for every $e\not= x\in G,\ 
\mbox{C}_G(x)$ is abelian. Prove that if $G$ is a $CA$-group, then
\begin{enumerate}
\item the relation $x\sim y$ if and only if 
$xy=yx$ is an equivalence relation on $G^\#$;
\item If $\cal C$ is an equivalence class in $G^\#$, then $H=
\{e\}\cup\cal C$ is a subgroup of $G$.
\end{enumerate}

\item Let $R$ be a {\em u.f.d.} in which every prime ideal is maximal.
Prove that every prime ideal is principal.

\item Let $p$ be a prime and let $R=\{\frac{a}{b}\in {\Q}|\
p\not\kern -0.1em\vert\ \  b\}$.  If $M$ is the principal ideal in $R$ generated
 by $p$,
prove that $M$ is the {\em unique} maximal ideal in $R$. ({\em Hint:} Show
that any element not in $M$ is a unit in $R$.)

\item Let $V$ be a vector space over the field ${\F}$ and let 
$T:V\to V$ be an {\em idempotent} linear transformation: $T^2=T$.
Prove that if $W\subseteq V$ is a $T$-invariant subspace of $V$, then
there exists a $T$-invariant subspace $W'\subseteq V$ such that
$V=W\oplus W'$.


\item Let $0\rightarrow M'\buildrel\mu\over\rightarrow 
M\buildrel\epsilon\over\rightarrow M''\rightarrow 0$ be a short exact 
sequences of $R$-modules, where $R$ is a ring. If $M''$ is $R$-free,
show that $M\cong \mu (M')\oplus M_0$, where $M_0\cong_RM''$.

\item Compute the Galois group over the rational field ${\Q}$ of 
the field ${\Q}(\zeta )$, where $\zeta = e^{2\pi i/12}$.

\item Compute the field extension degree $[{\Q}(\root\of{2}+\root\of{3})
:{\Q}]$.

\item Let ${\F}=\F_q$ be the finite field of $q$ elements, and let
$\K=\F_{q^2}\supseteq \F$ be a quadratic extension. Define the
{\em Frobenius automorphism} $F:\K\to \K$ by setting $F(\alpha )=\alpha^q,
\ \alpha\in\K$. If we define $N:\K^\times\to\F^\times $ by setting
$N(\alpha )=\alpha F(\alpha ),\ \alpha\in\K^\times$, show that $N$ is
a surjective homomorphism of groups whose kernel has order $q+1$. 



    
\end{enumerate}
\end{Large} \end{document}

















