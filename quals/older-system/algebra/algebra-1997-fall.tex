% latex file
\def\hcorrection#1{\advance\hoffset by #1 }
\def\vcorrection#1{\advance\voffset by #1 }

\documentclass{article}
\usepackage{my,amsxtra,amssymb,amsthm}

\vcorrection{-1.0in}
\hcorrection{-0.8in}
\textwidth 6.0in
\textheight 9.0in

\def\R{\mathbb R}
\def\C{\mathbb C}
\def\N{\mathbb N}
\def\Z{\mathbb Z}
\def\Q{\mathbb Q}

\begin{document}
%\begin{Large}






\begin{center}\begin{LARGE}
{\bf Algebra Qualifying Exam}\\ 
{\bf Fall 1997}\\ \end{LARGE}
\end{center}
\vspace{0.1in}
\noindent\hrulefill\\
All rings in this exam are associative with 1 and all integral domains
are commutative.

\begin{description}

\item[1.]
Let $\Bbb Q$ be the field of rational numbers. Show that the additive group
$A=(\Bbb Q, +)$ does not have any proper characteristic subgroups. Can you
generalize the above result when $\Bbb Q$ is any division ring? If yes, please
give a proof. If not, please give a counter example.

\item[2.]
Let $G$ a finite group. If, for any prime $p$ and any $p$-subgroup
$H$ of $G$, $N_G(H)$ has index at most 2 in $G$, show that $G$ has to be
a nilpotent group.

\item[3.]
Let $S$ be an integral domain. We say that an integral domain $R$ is
an {\it integral extension} of $S$ if $R \supseteq S$ and for each
$r \in R$ there exists a monic polynomial
$f(x) = x^n +s_1x^{n-1} + \dots + s_n \in S[x]$
with coefficients $s_i \in S$ and $n \geq 1$ such that $f(r) = 0$.

\item[\quad] (a)
Show that if $R$ is an integral extension of a field $S$, then
$R$ is also a field.

\item[\quad] (b)
If an integral extension $R$ of the integral domain $S$ is a field, is $S$
necessarily a field? Justify your answer.

\item[4.]
Let $S$ be the set of all nonzero integers which are sums of two squares of
integers.

\item[\quad] (a)
Show that $S$ is closed under multiplication.

\item[\quad] (b)
Show that the subring
$\Bbb Z_S = \left\{\frac{n}{s} |n \in \Bbb Z, s \in S \right\}$
of the set of rational numbers is the entire field $\Bbb Q$ of rational
numbers.

\item[5.]
Let $R$ be a ring and $M$ be a left $R$-module.

\item[\quad] (a)
Define what it means for $M$ to be {\it irreducible}.

\item[\quad] (b)
Define what it means for $M$ to be {\it indecomposable}.

\item[\quad] (c)
Describe, as complete as you can, the relations between the irreducibility
and indecomposability of an $R$-module $M$ by proving or giving
examples to your conclusions.

\item[6.]
Let $F$ be a field and $R=F[t] / <T^5>$. Describe all finitely generated
indecomposable $R$-modules up to $R$-module isomorphisms. (Hint:
$R$-modules are $F[T]$-modules. But what kind of $F[T]$-modules
are $R$-modules?)

\item[7.]
Suppose that $K$ is a splitting field of the polynomial $x^4-x^5$ over the
field $\Bbb Q$ of rational numbers. Compute $[K:\Bbb Q]$ and correctly
justify your answer.

\item[8.]
Let $E \supseteq F$ be a finite Galois extension of fields. Suppose that the
Galois group Gal$(E/F)$ is Abelian; show that for any $\alpha \in E$,
$F(\alpha)$ is splitting field of a polynomial in $F[x]$.

\item[9.]
Let $V$ be a vector space over the field $\Bbb Q$ of rational numbers and
$T:V \to V$ a linear transformation with the following invariant factors:
$$x +1, \quad x^2-1, \quad x^4 -1.$$

\item[\quad] (a)
Find $\dim V$.

\item[\quad] (b)
Find the nullity of $T$.

\item[\quad] (c)
Find the dimension of the subspace of all vectors fixed by $T$.

\item[10.]
Let $G$ be a solvable group and $K \neq \{1\}$ be a minimal finite normal
subgroup of $G$. Show that there exists a prime number $p$ and a positive
integer $r$ such that $K$ is isomorphic to the additive group of the vector
space $\Bbb F^r_p$ over the finite field $\Bbb F_p$ of $p$ elements.




\end{description}    
%\end{Large}
\end{document}














