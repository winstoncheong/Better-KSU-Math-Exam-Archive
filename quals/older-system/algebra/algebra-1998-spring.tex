%From dbski@math.ksu.edu Wed Jan 28 09:02:18 1998
%Subject: algebra qualifying exam (S98)




% latex file
\def\hcorrection#1{\advance\hoffset by #1 }
\def\vcorrection#1{\advance\voffset by #1 }
\input amssym.def
\input amssym
%\documentstyle{article}
\documentclass{article}
\pagestyle{plain}


\vcorrection{-1.0in}
\hcorrection{-0.8in}
\textwidth 6.0in
\textheight 9.0in
\begin{document}
\begin{Large}
\newcommand{\C}{{\Bbb C}}
\newcommand{\Q}{{\Bbb Q}}
\newcommand{\Gal}{\mbox{\rm Gal}}
\newcommand{\Aut}{\mbox{\rm Aut}}


\begin{center}\begin{LARGE}
{\bf  Algebra Qualifying Exam}\\ 
{\bf January 27, 1998}\\ \end{LARGE}
\end{center}
\vspace{0.1in}
\noindent\hrulefill\\
{\bf Instructions:} You are given 10 problems from which you are to do 8.
 Please indicate those  8 problems which you would like  to be graded 
by circling the problem numbers on the  problem sheet. 
{\bf Note:} All rings in this exam are associative and with 1 and 
all integral domains are commutative.
% $\Bbb Q $ and $ \Bbb C $ are
%the fields of  rational and complex  numbers respectively.

%$\Bbb Z $ and $ \Bbb Q $ are
%the sets of the integers and rational numbers respectively.

\vspace{0.2in}

\begin{enumerate}

%\item Let $G$ be a finite group of order 120. Show that either
%\begin{enumerate}
%\item There exists a nonidentity element normalizing every 5-Sylow
%subgroup of $G$, or
%\item $G\cong S_5$, the symmetric group on 5 letters.
%\end{enumerate}

\item Let $G$ be a finite group of even order. Prove $G$ contains
an odd number of involutions. (An {\em involution} is an element
of order 2.)

\item Let $G$ be a group and assume that for all $1\not= x\in G$,
$C_G(x)$ is abelian. Define a relation $\sim $ on $G$ by $x\sim y$
if and only if $xy=yx$. 
\begin{enumerate}
\item Prove that $\sim$ is an equivalence relation on the nonidentity
elements $G^\sharp $ of $G$.
\item If $\cal C$ is an equivalence class in $G^\sharp $, prove that
$\{1\}\cup \cal C $ is a subgroup of $G$.
\end{enumerate}

\item Let $G$ be a finite group acting transitively on the set $X$. Fix
$x\in X$, set $G_x=\mbox{Stab}_G(x)$ and let $P\in\mbox{Syl}_p(G_x)$,
where $p$ is prime. Prove that $N_G(P)$ acts transitively on $\mbox{Fix}(P)$.

\item Let ${\Bbb F}$ be a field and form the quotient ring
$$R\ =\ {\Bbb F}[x,y]/(x^2+y^2){\Bbb F}[x,y]$$
where $x$ and $y$ are indeterminate over ${\Bbb F}$. Prove that
$R$ is an integral domain if and only if $-1$ is not a square in 
${\Bbb F}$.

\item Let $R$ be a ring and let $M$ be an irreducible $R$-module. Prove
that if $0\not= m\in M$, then the set
$$ \mbox{Ann}_R(m)\ =\ \{r\in R|\ rm=0\}$$
is a maximal left ideal in $R$.

\item Suppose that $V$ is a vector space over the field ${\Bbb F}$ and
that $T:V\to V$ is a linear transformation having invariant factors
$$x-2,x(x-2),x^2(x-2)^2,x^2(x-2)^2(x^3-1).$$
\begin{enumerate}
\item What is the nullity of $T$ if $\mbox{char }{\Bbb F}\not= 2$?
\item What is the nullity of $T$ if $\mbox{char }{\Bbb F}= 2$?
\item What is the dimension of the set of vectors fixed by $T$ if
$\mbox{char }{\Bbb F}\not= 3$?
\item What is the dimension of the set of vectors fixed by $T$ if
$\mbox{char }{\Bbb F}= 3$?
\end{enumerate}

\item Let $M$ be a Noetherian $R$-module and let $\phi :M\to M$ be a
surjective $R$-module homomorphism. Prove that $\phi $ is injective.
(Hint: consider the increasing sequence of submodules $0\subseteq
\mbox{ker }\phi\subseteq \mbox{ker }\phi^2\subseteq\cdots $.)

\item Let $R$ be a unique factorization domain in which every prime
ideal is maximal. Prove that every prime ideal is a principal ideal.
(In fact, it turns out that {\em every} ideal is principal, but you
don't need to show this.)

\item Let ${\Bbb F}$ be a field and let $f(x), g(x)\in {\Bbb F}[x]$. 
Suppose that $f(x)$ and $g(x)$ share a common root in some extension
field  ${\Bbb K}\supseteq {\Bbb F}$. Show that the greatest common divisor
of $f(x)$ and $g(x)$  in ${\Bbb F}[x]$ cannot be a unit. 

%\item Let $p$ be a prime and let ${\Bbb F}$ be a field of characteristic
%$p$. Define the {\em Frobenius map} $\sigma :{\Bbb F}\to {\Bbb F}$
%by setting $\sigma (\alpha )=\alpha ^p$. 
%\begin{enumerate}
%\item Show that $\sigma :{\Bbb F}\to {\Bbb F}$ is an injective field
%endomorphism.
%\item Show by example that $\sigma $ need not be surjective.
%\item Compute the structure of $\mbox{inv}_{\Bbb F}(\sigma )$.
%\end{enumerate}

\item Let $p$ be prime and let $\zeta = e^{2\pi i/p}$. Compute the
degree of $m_{\zeta +\zeta^{-1}}(x)$. 
      


    
\end{enumerate}
\end{Large}
\end{document}
















