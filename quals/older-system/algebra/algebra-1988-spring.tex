% latex file
\def\hcorrection#1{\advance\hoffset by #1 }
\def\vcorrection#1{\advance\voffset by #1 }

\documentclass{article}
\usepackage{my,amsxtra,amssymb,amsthm}

\vcorrection{-1.0in}
\hcorrection{-0.8in}
\textwidth 6.0in
\textheight 9.0in

\def\R{\mathbb R}
\def\C{\mathbb C}
\def\N{\mathbb N}
\def\Z{\mathbb Z}
\def\Q{\mathbb Q}
\def\F{\mathbb F}

\begin{document}
%\begin{Large}






\begin{center}\begin{LARGE}
{\bf Algebra Qualifying Exam}\\ 
{\bf Spring 1988}\\ \end{LARGE}
\end{center}
\vspace{0.1in}
\noindent\hrulefill\\

Instruction: You are to do two problems from each of the four major-areas
({\bf Group Theory, Linear Algebra, Field and Galois Theory}, and {\bf Rings
and Modules}). Even if you attempt to do more, you will only be graded on
two problems from each section. For the graders' convenience, please circle
the problem numbers of those problems that you do from each of the four
major areas:

\centerline{{\bf I. Group Theory}}

\begin{description}
\item[1.]
Let $P$ be a finite $p$-group, where $p$ is a prime.

\item[\quad] (a)
Assume that $P$ acts on the finite set $\Omega$, with $p \nmid |\Omega|$.
Prove that there exists $\omega \in \Omega$ with $g(\omega) = \omega$ for
all $g \in P$.

\item[\quad] (b)
Explain how (a) proves that $Z(P) \neq \{1\}$ for any nontrivial finite
$p$-group $P$.

\item[2.]
Let $G$ be a finite group of order 120. Prove that $G$ cannot be a simple
group.

\item[3.]
Let
$$G= \left\{ \left[ \begin{matrix} a&b \\ 0&a^{-1} \end{matrix} \right]
  | a,b \in {\bf F}_q, a \neq 0 \right\}, $$
where ${\bf F}_q$ is the finite field of order $q$. Prove that $G$ is
solvable. (Hint: prove that $G$ is an extension of an abelian group by an
abelian group.)

\item[4.]
Let $A$ be the abelian group with presentation
$$A= < a_1, a_2, a_3 | -
  \begin{matrix}
        2a_1 & - & a_2 &&& = & 0 & \\
        a_1 & + & 2a_2 & - & 2a_3 & = & 0 & > \\
        & - & a_2 & + & 2a_3 & = & 0 &
        \end{matrix}.$$
Compute the {\it invariant factors} and, hence, the structure of $A$.

\centerline{{\bf II. Linear Algegra}}

\item[1.]
Let ${\bf F}$ be a field, let $V$ be an ${\bf F}$-vector space, and let
$T \in \hbox{End}_{\bf F} (V)$. Assume that $m_T(x) = f(x) g(x)$, where
$f(x), g(x) \in {\bf F} [x]$, and are {\it relatively prime}, and where
$m_T(x)$ is the minimal polynomial of $T$. Prove that
$V = \ker f(T) \oplus \ker g(T)$.

\item[2.]
Let $V$ be a complex vector space and let $T \in \hbox{End}_C (V)$. If
$T^k = 1_V$ for some integer $k>0$, prove that $T$ is diagonalizable.

\item[3.]
Let $A \in M_3({\bf Q})$ be the matrix
$$A = \left[ \begin{matrix}
        5&3&-3 \\
        3&5&-3 \\
        9&9&-7
        \end{matrix} \right].$$
Compute the invariant factors of $A$ and hence the {\it rational canonical
form} of $A$.

\item[4.]
Let ${\bf F}$ be a field, let $V$ be a finite-dimensional ${\bf F}$-vector
space, and let $T \in \hbox{End}_{\bf F} (V)$. Assume that the
minimal polynomial
$p(x) = m_T(x)$ of $T$ is prime, and set ${\bf K} = {\bf F}[x] / (p(x))$.

\item[\quad] (i)
For each $f(x) \in {\bf F}[x]$, denote by $\overline f (x)$ the element
$f(x) + (p(x)) \in {\bf K}$. Show that the scalar multiplication
$\overline f (x) \cdot v := f(T) (v), f(x) \in {\bf F}[x], v \in V$, is a well
{\it well-defined} ${\bf K}$-scalar multiplication on $V$, making $V$ into a
$K$-vector space.

\item[\quad] (ii)
Compute $\dim_{\bf K} (V)$ in terms of $n=\dim_{\bf F} (V)$ and
$d =\deg p(x)$.

\item[\quad] (iii)
Show that every ${\bf K}$-subspace of $V$ is a $T$-invariant ${\bf F}$-subspace of
$V$.

\item[\quad] (iv)
Conclude from the above that $T$ is {\it semisimple} in the sense that
every $T$-invariant subspace of $V$ has a $T$-invariant complement.

\centerline{{\bf Field and Galois Theory}}

\item[1.]
Let ${\bf F}$ be a subfield of the real field ${\bf R}$, and assume that
$\sqrt{\alpha} \in {\bf F}$ for each $0 \leq \alpha \in {\bf F}$. Prove that
$[{\bf F} : {\bf Q}] = \infty$, where ${\bf Q}$ is the rational field.

\item[2.]
Let $n \geq 3$ be an integer and let $f(x) = x^n - 2 \in {\bf Q} [x]$. Prove
that the Galois group of $f(x)$ is nonabelian.

\item[3.]
Let $p$ be a prime. Prove that for any positive integer $n$ there is an
irreducible polynomial $f(x) \in {\bf F}_p [x]$ of degree $n$.

\item[4.]
Let $f(x) = x^3 + x^2 - 2x - 1 \in {\bf Q} [x]$.

\item[\quad] (a)
Prove that $f(x)$ is irreducible.

\item[\quad] (b)
Prove that if $\alpha \in {\bf C}$ is a root of $f(x)$, so is $\alpha^2 - 2$.

\item[\quad] (c)
Using the above, compute the Galois group of $f(x)$.

\centerline{{\bf Rings and Modules}}

\item[1.]
Let $R$ be a ring and let $M$ be a Noetherian left $R$-module. If
$f \in \hbox{End}_R (M)$ is surjective, prove that $f$ is also injective.

\item[2.]
For each part below, give an appropriate example; {\it no proofs are
required}.

\item[\quad] (a)
A commutative ring $R$ and prime ideal $P \subseteq R$ such that $P$ is not
a maximal ideal of $R$.

\item[\quad] (b)
A commutative ring $R$ such that $R$ isn't a P.I.D. but every prime ideal
is maximal.

\item[\quad] (c)
A commutative integral domain $R$ which isn't a U.F.D.

\item[\quad] (d)
A non-commutative ring $R$ in which every left $R$-module is projective.

\item[\quad] (e)
A commutative semisimple Artinian ring.

\item[\quad] (f)
A non-commutative non-semisimple ring.

\item[3.]
Let $R$ be a ring and let $M$ be a left $R$-module. If $e \in R$ is an
idempotent (i.e. $e^2 = e$), prove that $M = eM \oplus (1-e)M$.

\item[4.]
Let ${\bf F}$ be a field and let $R$ be the ring
$$R = \left\{ \left[ \begin{matrix} a&0 \\ 0&b \end{matrix} \right] |
  a, b \in {\bf F} \right\}.$$

Define the $R$-modules
$$M_1 = \left\{ \left[ \begin{matrix} a \\ 0 \end{matrix} \right] |
  a \in {\bf F} \right\}, \quad M_2 = \left\{ \left[
  \begin{matrix} 0 \\ b \end{matrix} \right] | b \in {\bf F} \right\}.$$

Prove that $M_1 \ncong_R M_2$.


\end{description}    
%\end{Large}
\end{document}














