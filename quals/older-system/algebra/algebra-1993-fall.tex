% latex file
\def\hcorrection#1{\advance\hoffset by #1 }
\def\vcorrection#1{\advance\voffset by #1 }

\documentclass{article}
\usepackage{my,amsxtra,amssymb,amsthm}

\vcorrection{-1.0in}
\hcorrection{-0.8in}
\textwidth 6.0in
\textheight 9.0in

\def\R{\mathbb R}
\def\C{\mathbb C}
\def\N{\mathbb N}
\def\Z{\mathbb Z}
\def\Q{\mathbb Q}

\begin{document}
%\begin{Large}



\begin{center}\begin{LARGE}
{\bf Algebra Qualifying Exam}\\ 
{\bf Fall 1993}\\ \end{LARGE}
\end{center}
\vspace{0.1in}
\noindent\hrulefill\\
All rings are assumed to have a multiplicative identity, denoted 1. The
fields $\Q$, $\R$ and $\C$ are the fields of {\it rational, real} and
{\it complex} numbers, respectively.

\begin{description}

\item[1.]
Let $p$ be an odd prime. If the congruence $x^2 \equiv -1 (\mod p)$
has a solution, show that $p \equiv 1(\mod 4)$.

\item[2.]
Prove, or give a counterexample to the assertion that any torsion-free
abelian group is free.

\item[3.]
Let $G$ be a group of order $2p$, where $p$ is an odd prime. Suppose that
$G$ has a normal Sylow 2-subgroup. Show that $G$ is cyclic.

\item[4.]
Prove, or give a counterexample.

\item[\quad] (a)
Each ideal of $Z[x]$ is principal.

\item[\quad] (b)
If $I$ is a maximal ideal of $Z$, then $I[x]$ is maximal ideal of
$Z[x]$. Here, $I[x]$ is the ideal of $Z[x]$ consisting of polynomials
with  coefficients in $I$.

\item[5.]
Consider the ring $R=Z[\sqrt{5}] = \{a + b \sqrt{5} | a,b \in Z\}$.
Show that the element $3 \in R$ is irreducible but not prime. (Hint: Note
that $3|(4+ \sqrt{5}) (4 - \sqrt{5})$.)

\item[6.]
Let $f(x) = x^4 + 1$. Is $f(x)$ irreducible over

\item[\quad] (a)
$\R$?

\item[\quad] (b)
$\Q$?

\item[\quad] (c)
$\C$?

\item[\quad] (d)
$F_{16}$? (Finite field of 16 elements.)

\item[\quad] (e)
$F_7$? (Finite field of 7 elements.)

\item[7.]
Let $f(x)$ be an irreducible polynomial of degree 3 in $\Q[x]$, and assume
that $f(x)$ has a non-real root. Prove that if $K$ is a splitting
field over $\Q$ for $f(x)$, then $Gal(K/\Q) \cong S_3$.

\item[8.]
Give as long a list as possible of square matrices

\item[\quad] (a)
Each matrix has characteristic polynomial $(x-2)^4(x-3)$.

\item[\quad] (b)
Each matrix has minimal polynomial $(x-2)^2(x-3)$.

\item[\quad] (c)
No two matrices on the list are similar.

\item[9.]
Let $V$ be a $n$-dimensional vector space over the complex field $C$.
Assume that $S,T : V \to V$ are linear transformations such that
$ST = TS$. Show that $T$ and $S$ have a common eigenvector in $V$. Must
they also have a common eigenvalue?





\end{description}    
%\end{Large}
\end{document}














