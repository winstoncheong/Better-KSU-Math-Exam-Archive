% latex file
\def\hcorrection#1{\advance\hoffset by #1 }
\def\vcorrection#1{\advance\voffset by #1 }

\documentclass{article}
\usepackage{my,amsxtra,amssymb,amsthm}

\vcorrection{-1.0in}
\hcorrection{-0.8in}
\textwidth 6.0in
\textheight 9.0in

\def\R{\mathbb R}
\def\C{\mathbb C}
\def\N{\mathbb N}
\def\Z{\mathbb Z}
\def\Q{\mathbb Q}


\begin{document}
%\begin{Large}



\begin{center}\begin{LARGE}
{\bf Algebra Qualifying Exam}\\ 
{\bf Spring 1991}\\ \end{LARGE}
\end{center}
\vspace{0.1in}
\noindent\hrulefill\\
All rings are assumed to have a multiplicative identity, denoted 1. The
fields $\Q$, $\R$ and $\C$ are the fields of {\it rational, real} and
{\it complex} numbers, respectively.

\begin{description}

\item[1.]
Let $G$ be a finite group, $N$ a normal subgroup of $G$, and let
$g \in G-N$. If $p$ is a prime with $g^p \in N$, prove that the cyclic
group $<g>$ has a subgroup of order $p$.

\item[2.]
If $p$ is prime, prove that the center of any non-identity finite
$p$-group is non-trivial.

\item[3.]
Prove that any simple group of order 60 is isomorphic to the alternating
group $A_5$.

\item[4.]
Let $R$ be the polynomial ring $\Z[x]$, and let $M$ be the ideal in $R$
generated by the elements $2, x \in R$. Prove that $M$ is a maximal ideal
in $R$.

\item[5.]
Let $p$ be a prime and let $R = \left\{\frac{a}{b} \in \Q | p \nmid b \right\}$.
If $M$ is the principal ideal in $R$ generated by $p$, prove that
$M$ is the {\it unique} maximal ideal in $R$. ({\it Hint:} Show that
any element not in $M$ is a unit in $R$.)

\item[6.]
Let $f(x) = x^5-2 \in \Q[x]$, and let $\omega$ be a complex primitive fifth
root of unity. Show that $\Q(\omega, \sqrt[5]{2})$ is a splitting
field for $f(x)$.

\item[7.]
Let $f(x) = x^5 - 1 \in \Q[x]$. Prove that the Galois group of $F(x)$ over
$\Q$ is nonabelian.

\item[8.]
Let $V$ be an $n$-dimensional vector space over a field $F$, and let
${\cal B} = \{x_1, x_2, \dots, x_n\}$ be a basis for $V$. Let $V^\ast$
denote the dual space of $V$, that is, $V^\ast$ is the vector space
$Hom_F(V,F)$ of all linear transformations $\lambda : V \to F$.
Define elements $\lambda_1, \dots, \lambda_n$ of $V^\ast$ by setting
$$\lambda_i \left( \sum^n_{j=1} a_jx_j \right) = a_i,$$
$1 \leq i \leq n, a_j \in F$, and put
${\cal B}^\ast = \{\lambda_1, \dots, \lambda_n\}$. Show that ${\cal B}^\ast$
is a basis of $V^\ast$.

\item[9.]
Let $V$ be an $n$-dimensional vector space over a field $F$, and let
$$V= V_0 \supseteq V_1 \supseteq \dots \supseteq V_n = 0$$
be a chain of subspaces of $V$, with $\dim(V_i/ V_{i+1}) = 1$ for
$i=0,1, \dots, n-1$. Suppose that $T:V \to V$ is linear transformation
satisfying $T(V_i) \subseteq V_{i+1}$ for all $i = 0,1, \dots, n-1$.
Compute the characteristic polynomial of $T$.






\end{description}    
%\end{Large}
\end{document}














