
% latex file
\def\hcorrection#1{\advance\hoffset by #1 }
\def\vcorrection#1{\advance\voffset by #1 }
\input amssym.def
\input amssym
\documentclass
{article}
\pagestyle{plain}

\def\nodivide{\not\kern -0.13em\vert \ }


\vcorrection{-1.0in}
\hcorrection{-0.8in}
\textwidth 6.0in
\textheight 9.0in
\begin{document}
\begin{Large}
\newcommand{\C}{{\Bbb C}}
\newcommand{\Q}{{\Bbb Q}}
\newcommand{\Z}{{\Bbb Z}}
\newcommand{\N}{{\Bbb N}}
\newcommand{\F}{{\Bbb F}}
\newcommand{\K}{{\Bbb K}}
\newcommand{\E}{{\Bbb E}}

\newcommand{\Gal}{{\rm Gal}}
\newcommand{\Aut}{{\rm Aut}}



\begin{center}\begin{LARGE}
{\bf  Algebra Qualifying Exam}\\ 
{\bf January 20, 2000}\\ \end{LARGE}
\end{center}
\vspace{0.1in}
\noindent\hrulefill\\
{\bf Instructions:} You are given 10 problems from which you are to do 8.
 Please indicate those  8 problems that you would like  to be graded 
by circling the problem numbers on the  problem sheet. 
{\bf Note:} All rings in this exam are associative and with 1 and 
all integral domains are commutative.
%$\Bbb Q $ and $ \Bbb C $ are
%the fields of  rational and complex  numbers respectively.

%$\Bbb Z $ and $ \Bbb Q $ are
%the sets of the integers and rational numbers respectively.

\vspace{0.2in}

\begin{enumerate}


\item Let $G$ be a finite group, let $p$ be a prime,
and let $S$ be a Sylow $p$-subgroup
of $G$. Prove that if $N\triangleleft G$ is a normal subgroup
of $G$, then $S\cap N$ is a Sylow $p$-subgroup of $N$. 

\item Let $G$ be a finite group of odd order and let $x\in G$ be a
nonidentity element. Prove that $x$ and $x^{-1}$ are {\em not}
conjugate in $G$.

\item Let $G$ be a group acting transitively on the finite set $X$. 
Let $x\in X$ and denote by $G_x=\{g\in G|\ gx=x\}$ the 
{\em isotropy subgroup} of $x$. Let $H\triangleleft G$ be a normal
subgroup of $G$; note that $HG_x$ is a subgroup of $G$ and
that $H$ acts on $X$. Prove that the number of distinct orbits of
the action of $H$ on $X$ equals the index $[G:HG_x]$.


\item Let $I=(2,x)=\{2f(x)+xg(x)|\ f(x),g(x)\in\Z[x]\}$ be the ideal
in the ring $R=\Z[x]$ of polynomials in 
the indeterminate $x$ with integer coefficients. Prove that $I$ is {\em not}
a free $R$-module.


\item Let $R$ be a Euclidean domain with respect to the function
$d:R-\{0\}\to \Z^+\ (=\{1,2,\ldots \})$. Assume that $d$ satisfies
\begin{enumerate}
\item $d(ab)=d(a)d(b)$, for all $a,b\in R-\{0\},$
\item $d(a+b)\leq {\rm max}\{d(a),d(b)\}$, for all $a,b,a+b\in R-\{0\}.$
\end{enumerate}
Prove that either $R$ is a field, or that there exists a field
$\F\subseteq R$ such that $R\cong \F[x]$, the ring of polynomials
in the indeterminate $x$ with coefficients in $\F$.
[Hint:
Let $\F=\{a\in R|\ d(a)=1\}$.]

\item Let $A,B:V\to V$ be linear transformations on the finite dimensional
vector space $V$ over the complex numbers $\C$. Prove that if $AB=BA$ 
then 
there exists a nonzero vector $0\not=v\in V$ that is simultaneously an
eigenvector for both $A$ and $B$.


\item Let $T:V\to V$ be a linear transformation of the finite
dimensional vector space $V$ over the field $\F$. Define the
usual $\F[x]$-module structure on $V$ by setting $f(x)\cdot v
=f(T)(v)$, $f(x)\in\F[x],\ v\in V$. Prove that $V$ is a cyclic
$\F[x]$-module if and only if the characteristic polynomial of
$T$ equals the minimal polynomial of $T$.


\item Let ${\Bbb F}={\Bbb F}_q$ be the finite field of 
$q\ (=p^r)$ elements, where
$p$ is prime, and let ${\Bbb K}={\Bbb F}_{q^4}\supseteq {\Bbb F}$.
Say that elements $\alpha ,\beta\in {\Bbb K}$ are {\em equivalent} if 
they have the same minimimal polynomial over ${\Bbb F}$. Clearly
this is an equivalence relation on ${\Bbb K}$. Compute the number of 
equivalence classes in ${\Bbb K}$ as a function of $q$. (Hint: 
consider $\F\subseteq\F_{q^2}\subseteq\F_{q^4}=\K$.)
         
\item Let $\F\subseteq \E\subseteq \K$ be fields, where the extension
degrees are finite, 
let $\alpha\in \K$, and let $f(x)$ be the minimal polynomial of
$\alpha $ over $\F$.  Assume that $[\E:\F]$ and
$\mbox{deg }f(x)$ are relatively prime. Prove that $f(x)$ is also the
minimal polynomial of $\alpha $ over $\E$.

\item Let $n\geq 3$ be an integer and let $f(x)=x^n-2\in \Q[x]$.
Prove that the Galois group of $f(x)$ is nonabelian but solvable.


\end{enumerate}
\end{Large}
\end{document}















