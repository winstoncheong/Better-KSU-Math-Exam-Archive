% latex file
\def\hcorrection#1{\advance\hoffset by #1 }
\def\vcorrection#1{\advance\voffset by #1 }

\documentclass{article}
\usepackage{my,amsxtra,amssymb,amsthm}

\vcorrection{-1.0in}
\hcorrection{-0.8in}
\textwidth 6.0in
\textheight 9.0in

\def\R{\mathbb R}
\def\C{\mathbb C}
\def\N{\mathbb N}
\def\Z{\mathbb Z}
\def\Q{\mathbb Q}

\begin{document}
%\begin{Large}






\begin{center}\begin{LARGE}
{\bf Algebra Qualifying Exam}\\ 
{\bf Fall 1994}\\ \end{LARGE}
\end{center}
\vspace{0.1in}
\noindent\hrulefill\\
All rings are assumed to have a multiplicative identity, denoted 1. The
fields $\Q$, $\R$ and $\C$ are the fields of {\it rational, real} and
{\it complex} numbers, respectively.

\begin{description}

\item[1.]
Let $P$ be a finite $p$-group for a given prime number $p$. For any
$x \in P$ show that either $\langle x \rangle$ is a normal subgroup of
$P$, or there
exists $g \in P$ such that $[x, gxg^{-1}] = 1$ and
$x \neq gxg^{-1}$.

\item[2.]
Let $G$ be a finite groups and $p$ a prime number dividing the order of $G$.
$Syl_p (G)$ is the set of all Sylow $p$-subgroups of $G$.

\item[\quad] (a)
Show that $N_P (Q) = P \cap Q$ for $P,Q \in Syl_p (G)$. Here
$N_P(Q) = \{x \in P |xQx^{-1} = Q\}$
is the normalizer of $Q$ in $P$.

\item[\quad] (b)
Show that is there is $P \in Syl_p (G)$ such that $P \cap xPx^{-1} = P$
or $\{1\}$ for any $x \in G$, then $|Syl_p (G) | \equiv 1$
(mod $|P|$).


\item[3.]
Prove that every prime ideal in a PID is maximal. Then give an example of an
integral domain, in which there is a prime ideal which is not maximal.

\item[4.]
Recall that a ring is called Noetherian if every ascending chain of ideals
terminates. Show that any PID is Noetherian. Give an example of a
Noetherian integral domain which is not a PID.

\item[5.]
Let $F$ by any field and $f(x) \in F[x]$. If $E$ is the splitting field of
$f(x)$ over $F$, show that $[E:F] \leq (\deg(f(x))!$. Give an example
for which the equality holds.

\item[6.]
Let $\alpha = \sqrt{3} + \sqrt[3]{2}$.
It is known that
$[\Bbb Q (\alpha) : \Bbb Q] = 6$.

\item[\quad] (a)
Calculate the minimal polynomial of $\alpha$ over the field $\Bbb Q$ of
rational numbers.

\item[\quad] (b)
Is the field $\Bbb Q(\alpha)$ Galois over $\Bbb Q$? If yes, determine the
Galois group. If not, justify your answer.

\item[7.]
Let $R$ be a ring. If $M$ is a left $R$-module, show that for any
$R$-submodule $N$ of $M$, the set $\{x \in R |x M \subseteq N \}$ is a
two-sided ideal of $R$.

\item[8.]
Let $R$ be a ring. Prove Schur's Lemma: For any two simple left
$R$-modules $M$ and $N$, any $R$-module homomorphism $\phi: M \to N$
is either identically zero or an isomorphism.

\item[9.]
Let $F$ by any fixed field and $V$ a vector space over $F$ (of any dimension).
Suppose that $A:V \to V$ is a nilpotent linear transformation
(i.e., $A^n = 0$, for some $n$). Show that $A$ has at least
one eigenvector with eigenvalue in $F$.

\item[10.]
Let
$A= \begin{bmatrix}
        5&1&2 \\
        0&4&-1 \\
        0&0&4
        \end{bmatrix}$
be a $3 \times 3$ matrix. Show that there exist complex matrices $D$ and $N$
such that $D$ is diagonalizable, $N$ is nilpotenet, $DN=ND$, and $A=D+N$.





\end{description}    
%\end{Large}
\end{document}














