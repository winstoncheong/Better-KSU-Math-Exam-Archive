% latex file
\def\hcorrection#1{\advance\hoffset by #1 }
\def\vcorrection#1{\advance\voffset by #1 }

\documentclass{article}
\usepackage{my,amsxtra,amssymb,amsthm}

\vcorrection{-1.0in}
\hcorrection{-0.8in}
\textwidth 6.0in
\textheight 9.0in
\begin{document}
%\begin{Large}






\begin{center}\begin{LARGE}
{\bf Complex Analysis Qualifying Exam}\\ 
{\bf Fall 1992}\\ \end{LARGE}
\end{center}
\vspace{0.1in}
\noindent\hrulefill\\
In the following $\Bbb C$ denotes the set of all complex numbers, $\Bbb D$
the set $\{z \in \Bbb C : |z| < 1\}$ and $H(\Bbb D)$ the set of all
holomorphic functions on $\Bbb D$.

\begin{description}
\item[1.]
Show that if $g: \Omega \hbox{\ open} \subset \Bbb C \to \Bbb C$
is continuous and $e^g$ is holomorphic,
then $g$ is holomorphic. (That is, a {\bf continuous} logarithm of a
holomorphic function is necessarliy holomorphic.)

\item[2.]
Show directly (without reference to the concept of simple-connectivity) that
every zero-free function $f \in H(\Bbb D)$ has a holomorphic logarithm;
that is, $\exists g \in H(\Bbb D)$ such that $f = e^g$.

\item[3.]
Show directly (without reference to the concept of simple-connectivity) that
the identity function, $I(z) = z$, in $\Bbb C \backslash \{0\}$ has no
continuous logarithm. {\bf HINT:} Problem 1 may be useful.

\item[4.] (a)
$f$ is continuous on $\overline{\Bbb D}$, holomorphic in $\Bbb D$. Show that
$f$ is uniformly approximable on $\overline{\Bbb D}$ by polynomials.
{\bf HINT:} First approximate $f$ uniformly on $\overline{\Bbb D}$ by a
function $f_r$ which is holomorphic in $D(0,1/r), 0 < r< 1$.

\item[\quad] (b)
State and prove the converse of (a).

\item[5.]
State

\item[\quad] (a)
the Maximum Modulus Principle for holomorphic functions,

\item[\quad] (b)
the Open Map Theorem for holomorphic functions.

\item[\quad] (c)
Show that (a) can be deduced from (b).

\item[6.]
Show that $\int_{\partial \Bbb D} \frac{e^{\pi z}}{4z^2 + 1} dz = \pi i$.

\item[7.]
$f$ is holomorphic in $A:= \Bbb D \backslash \{0\}$ and satisfies
$|f(z)| < |z|^{3/2}$ for all $z \in A$. Show that
$|f(1/2)| \leq 1/4$. {\bf HINT:} First see if the function
$g(z) := f(z)/z$ can be holomorphically extended into $\Bbb D$. What will
its value at 0 have to be?

\item[8.]
$f$ is holomorphic and one-to-one in the region $\Omega$. Let $G =f(\Omega)$
and $g : G \to \Omega$ be the inverse of $f$. Prove that $g$ is holomorphic.
{\bf HINT:} You will need to prove {\bf en route} that $f^\prime$ is
zero-free in $\Omega$.

\item[9.] (a)
Define (don't just name) the three kinds of isolated singularity which
holomorphic functions can have and give an example of each.

\item[\quad] (b)
What does the Casorati-Weierstrass Theorem say about one of these kinds of
singularities?

\item[\quad] (c)
What does the Great Picard Theorem say about one of these kinds of
singularities?

\item[\quad] (d)
State Mittag-Leffler's Theorem regarding the principal parts of a meromorphic
function.





\end{description}    
%\end{Large}
\end{document}














