%Burckel/Bennett - F97 topology QE
% latex file
\def\hcorrection#1{\advance\hoffset by #1 }
\def\vcorrection#1{\advance\voffset by #1 }

\input amssym.def   %defines Bbb and gothic (frak)
\input amssym       %defines Bbb and gothic (frak)

%\documentstyle[bbb]{report}
\documentclass[bbb]{report}
\vcorrection{-0.8in}
\hcorrection{-1.3in}

\topmargin  0in
\textwidth 7.2in
\textheight 9.5in

\def\ds{\displaystyle}

\begin{document}

\pagestyle{empty}

\begin{Large}
\begin{center}
{\bf   COMPLEX VARIABLES QUALIFYING EXAM \\
   Fall 1997 \\
   (Burckel and Bennett)} \\
\end{center}
\end{Large}

\begin{large}

\vspace{.2in}

\begin{description}
{\bf

\item[1.]  
$f$ is holomorphic in ${\Bbb D}:=\{z\in{\Bbb C}:|z|<1\}$.

\item[\quad (i)] Show that 
$$F(z):=\overline{f(\overline{z})},\quad z\in{\Bbb D}$$
is also holomorphic in ${\Bbb D}$.

\item[\quad(ii)] Show that if $f$ is real-valued on 
${\Bbb D}\cap{\Bbb R}$, then
$$\overline{f(z)}=f(\overline{z})\qquad \forall z\in{\Bbb D}. $$

\item[\quad (iii)] If $g$ is holomorphic in all of ${\Bbb C}$ 
and real-valued on $[-1,1]$, does it follow as in (ii) that
$$ \overline{g(z)}=g(\overline{z})\qquad \forall z\in {\Bbb C}\quad? $$

\vspace{.5in}

\item[2.] $f$ is holomorphic and zero-free in the region $\Omega$, 
$z_0\in\Omega$, and for each $z\in\Omega$, $\gamma_z$ is a 
piecewise-smooth curve in $\Omega$ joining $z_0$ to $z$. A function 
$L_f$ is then well defined by
$$ L_f(z):=\int_{\gamma_z} f'/f, \qquad z\in\Omega. $$

\item[\quad (i)] Show that in case $\Omega={\Bbb D}$ this function 
is holomorphic and satisfies
$$ L'_f=f'/f. $$

\item[\quad (ii)] Infer that $f=f(z_0)e^{L_f}$.

\item[\quad (iii)] For what other $\Omega$ besides ${\Bbb D}$ does 
every zero-free holomorphic function $f$ on $\Omega$ satisfy (i) and (ii)?

\item[\quad (iv)] Show that in $\Omega:=\{z\in{\Bbb D}:1/2<|z|<2\}$ 
with $z_0:=1$, say, conclusion (i) may fail. Show in fact that for 
an appropriate zero-free holomorphic $f$, the function $f'/f$ has no 
primitive.

\item[\quad\quad] [It is an interesting fact, which you need not prove, 
that (ii) holds for every $\Omega$.]

\vspace{.5in}

\item[3.] 
Show that $\zeta(z):=\ds\sum^\infty_{n=1}n^{-z}$ converges and 
defines a holomorphic function in $\hbox{Re}(z)>1$.

\vspace{.5in}

\item[4.] 
Let $U$ be a domain in ${\Bbb C}$, ${\cal F}\subset C(U)$ an 
equicontinuous family, $(f_n)\subset {\cal F}$ a pointwise convergent 
sequence. Show that this sequence converges uniformly on each
compact subset of $U$.

\vspace{.5in}

\item[5.] 
Let ${\Bbb D}:=\{z\in{\Bbb C}:|z|<1\}$, $r$, $R$ positive real numbers.

\item[\quad (i)] State Schwarz' Lemma for holomorphic maps 
$f:{\Bbb D}\to{\Bbb D}$, and formulate a version for maps from
$r\cdot {\Bbb D}$ into $R\cdot{\Bbb D}$.

\item[\quad (ii)] Use (i) to prove Liouville's theorem about 
entire functions.

\vspace{.5in}

\item[6.] 
Show that $\ds\int^\infty_0\cos(t^2)\,dt=\ds\int^\infty_0\sin(t^2)\,dt
=\frac{\sqrt{2\pi}}{4}$.

\vspace{.5in}

{\it Hint:} Use the contour 

\vspace{.5in}


\item[7.] 
A {\it M\"obius transformation} is a mapping from the extended complex
plane into itself of the form 
$f(z):=\frac{az+b}{cz+d}\quad (a,b,c,d\in{\Bbb C},\ ad-bc\not=0)$.

\item[\quad (i)] Show that the set of all M\"obius transformations 
forms a group under composition.

\vspace{.25in}

A M\"obius transformation is called {\it hyperbolic} if it is 
conjugate to a {\it dilation} (i.e., to a $g$ of form 
$g(z)=rz$, $r>0$).

\item[\quad (ii)] Show that every hyperbolic transformation $f$ has 
two distinct fixed-points and that $f(C)=C$ for every circle $C$ 
that passes through both these fixed-points.





}
\end{description}

\end{large}

\end{document}
