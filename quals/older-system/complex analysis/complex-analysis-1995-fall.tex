% latex file
\def\hcorrection#1{\advance\hoffset by #1 }
\def\vcorrection#1{\advance\voffset by #1 }

\documentclass{article}
\usepackage{my,amsxtra,amssymb,amsthm}

\pagestyle{empty}

\vcorrection{-1.25in}
\hcorrection{-0.8in}
\textwidth 6.0in
\textheight 9.5in
\begin{document}


%\begin{Large}

\begin{center}\begin{LARGE}
{\bf Complex Variables Qualifying Exam}\\ 
{\bf Fall 1995}\\ \end{LARGE}
\end{center}
\vspace{0.1in}
\noindent\hrulefill\\
\begin{description}

\item[1.]
Find a function $f: \Bbb C \to \Bbb C$ such that $f$ is complex
differentiable exactly on the real axis.

Hint: Set $\frac{\partial f}{\partial \overline z} = z - \overline z$).

\item[2.]
Compute $\int^\infty_0 \frac{dx}{1+x^n}$ for all $n \geq 2$.

Hint: Use the contour below

\vspace{.25in}
\begin{center}
\setlength{\unitlength}{0.015in}
\begin{picture}(0,0)
\put(-10,0){\line(1,0){60}}\put(0,0){\vector(1,0){45}} %+x
\put(0,-10){\line(0,1){50}} %+y

\put(25,25){\line(-1,-1){25}}\put(25,25){\vector(-1,-1){15}}%-y-x
\put(3.75,0){\oval(7,7)[tr]}
\put(15,5){$\frac{2\pi}{n}$}

\put(42,15){\vector(-1,1){2}}
\qbezier[100](50,0)(40,25)(25,25)

\end{picture}
\end{center}

%\vspace{.1in}
\item[3.]
Suppose $f$ is entire and $|f(z)| \leq A + B|z|^k$ for some constants
$A,B, k > 0$. Show that $f$ is a polynomial.

\item[4.]
Suppose $f$ is an entire function and Range
$(f) \subset \{-1 < \hbox{Re}(z) < 1\}$.
Show $f$ is constant.

\item[5.]
Construct an entire function having simple zeros at the points
$\{a+ bi: a, b \in \Bbb Z\}$.

\item[6.]
Let $D = \{z : |z| < 1\}$ and let $f : D \to D$ be an analytic univalent
(one-to-one) function. Let $\Omega = f(D)$. If
$f(z) = \sum^\infty_{n=0} a_nz^n$ show that the area of $\Omega$ is equal
to $\pi \sum^\infty_{n=1} n|a_n|^2$. (Hint: Consider the Jacobian of
the mapping.)

\item[7.] (a)
Let $f(z) = \frac{P(z)}{Q(z)}$ where $P, Q$ are polynomials with
$\deg (Q) \geq 2 + \deg (P)$. Let $M = \{u_1, \dots, u_k\}$ be the poles
of $f$ and assume $M \cap \Bbb Z = \oslash$. Show that
$$\sum^\infty_{n=-\infty} f(n) = -\sum^k_{\ell = 1} \hbox{Res} (g; u_\ell)
  \quad \hbox{where\ } g(z) = \pi f(z) \cot(\pi z)$$

Hint: Consider the contour


\vspace{.25in}
\begin{center}
\setlength{\unitlength}{0.005in}
\begin{picture}(0,0)
\put(0,0){\line(1,0){125}} %+x
  \multiput(10,5)(10,0){2}{\line(0,-1){10}}
  \multiput(70,5)(10,0){2}{\line(0,-1){10}}
  \put(30,-15){\begin{tiny}$\dots n\quad n+1$\end{tiny}}

\put(0,0){\line(-1,0){125}}%-x
  \multiput(-10,5)(-10,0){2}{\line(0,-1){10}}
  \multiput(-70,5)(-10,0){2}{\line(0,-1){10}}
  \put(-65,-15){\begin{tiny}$\dots$\end{tiny}}

\put(0,0){\line(0,-1){125}}%-y
  \multiput(5,-10)(0,-10){2}{\line(-1,0){10}}
  \multiput(5,-70)(0,-10){2}{\line(-1,0){10}}

\put(0,0){\line(0,1){125}} %+y
  \multiput(5,10)(0,10){2}{\line(-1,0){10}}
  \multiput(5,70)(0,10){2}{\line(-1,0){10}}

\put(75,75){\line(-1,0){150}}
\put(85,85)
{\begin{tiny}$(n+\frac{1}{2})+(n+\frac{1}{2})i$\end{tiny}}
\put(75,75){\vector(-1,0){35}}

\put(-75,75){\line(0,-1){150}}
\put(-250,85)
{\begin{tiny}$-(n+\frac{1}{2})+(n+\frac{1}{2})i$\end{tiny}}
\put(-75,75){\vector(0,-1){35}}

\put(-75,-75){\line(1,0){150}}
\put(-250,-85)
{\begin{tiny}$-(n+\frac{1}{2})-(n+\frac{1}{2})i$\end{tiny}}
\put(-75,-75){\vector(1,0){35}}

\put(75,-75){\line(0,1){150}}
\put(85,-85)
{\begin{tiny}$(n+\frac{1}{2})-(n+\frac{1}{2})i$\end{tiny}}
\put(75,-75){\vector(0,1){105}}


\end{picture}
\end{center}

\vspace{.5in}

\item[\quad] (b)
Use a variation on (a) to compute $\sum^\infty_{n=1} \frac{1}{n^2}$.

\item[8.]
Let $u$ be a harmonic function in the first quadrant satisfying the
illustrated boundary conditions
$u(x,0) = \begin{cases} 1\ x< 1 \\ 0\ x \geq 1 \end{cases}
  u(0,y) =
  \begin{cases} 1\ y < 1 \\ 0\ y \geq 1 \end{cases}$


\vspace{1in}
\begin{center}
\setlength{\unitlength}{0.015in}
\begin{picture}(0,0)

\put(45,0){\line(1,0){30}} %+x
\multiput(0,0)(7,0){7}{\line(1,0){5}}
\put(-5,45){\line(1,0){10}}
\put(20,-10){$1$}
\put(60,-10){$0$}

\put(0,45){\line(0,1){30}} %+y
\multiput(0,0)(0,7){7}{\line(0,1){5}}
\put(45,5){\line(0,-1){10}}
\put(-10,20){$1$}
\put(-10,60){$0$}

\put(30,30){$\hbox{\begin{tiny}$\circ$\end{tiny}}\ u(1,1)+?$}
\end{picture}
\end{center}


Compute $u(1,1)$.





\end{description}    
%\end{Large}
\end{document}














