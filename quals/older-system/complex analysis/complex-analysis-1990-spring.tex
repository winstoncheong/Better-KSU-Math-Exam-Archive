% latex file
\def\hcorrection#1{\advance\hoffset by #1 }
\def\vcorrection#1{\advance\voffset by #1 }

\documentclass{article}
\usepackage{my,amsxtra,amssymb,amsthm}

\vcorrection{-1.0in}
\hcorrection{-0.8in}
\textwidth 6.0in
\textheight 9.0in
\begin{document}
%\begin{Large}






\begin{center}\begin{LARGE}
{\bf Complex Variables Qualifying Exam}\\ 
{\bf Spring 1990}\\ \end{LARGE}
\end{center}
\vspace{0.1in}
\noindent\hrulefill\\

In what follows $\Bbb R$ is the real numbers, $\Bbb C$ the complex numbers,
$\Bbb N$ the natural numbers, $\Bbb Z$ the integers, $\Bbb D$ the open
disc cented at 0 in $\Bbb C$, $\Bbb T$ the boundary of $\Bbb D$.

\begin{description}
\item[1.]
What is the ``reflection principle" for holomorphic functions? Prove the
simplest version of this principle, i.e., that involving reflection in the
real axis.

\item[2.]
Show that if $f$ is holomorphic and zero-free in the open set $U$, then
$|f|^p$ is subharmonic in $U$ for every real $p$. {\bf Hint:} If
$f = e^g$, the problem is easy. (For extra credit: Is the result true if
$f$ is permitted to have zeros?)

\item[3.]
Write an essay on the role of simple-connectivity in complex analysis. Touch
on the following points:

\item[\quad] (a)
A definition of simple-connectivity appropriate to regions in $\Bbb C$,

\item[\quad] (b)
several important equivalences of your definiton,

\item[\quad] (c)
connection with holomorphic logarithms and holomorphic roots of
holomorphic functions,

\item[\quad] (d)
Cauchy's Integral Theorem and existence of primitives,

\item[\quad] (e)
special role of the regions $\Bbb D$ and $\Bbb C$,

\item[\quad] (f)
existence of harmonic conjugates and solvability of the Dirichlet problem.

\item[4.]
$\phi$ is meromorphic in $\Bbb C$. Explain why (or why not) there must
exist entire functions $f$ and $g$ such that $\phi = \frac{f}{g}$.

\item[5.]
Describe pictorially the region
$\Omega := \Bbb C\ \backslash \{z : \hbox{Rez} = \hbox{Im}z \geq 1 \}$ and find explicitly a
conformal map of it onto $\Bbb D$.

\item[6.]
Suppose $f$ has a pole at $z_0$, that $0 \leq \alpha < \beta \leq 2\pi$
and $T_r (\alpha, \beta) := \{re^{i\theta} + z_0 : \alpha \leq \theta \leq
 \beta \}$.
Evaluate $\lim_{r \to 0} \int_{T_r(\alpha, \beta)} f(z)dz$ in terms of the
Laurent coefficients of $f$ at $z_0$.

\item[7.]
Let $f(z)$ denote any holomorphic square-root of $z$ in
$D_1 := \{z \in \Bbb C : |z-1| < 1 \}$ and let $F$ be an analytic continuation
of $f$ along a curve from 1 to -1. Show that $F(-1)$ is either $i$ or $-i$.

\item[8.]
Consider the (concentric) annulus $A := \{z \in \Bbb C : 1 < |z| < 2\}$.
What linear fractional transformations map $A$ to another annulus
$A^\ast := \{z \in \Bbb C : r < |z-a| < s\}$? {\bf Hint:} Build the map
as a composite of simpler ones.

\item[9.] (a)
Define: $z_0$ is an $n$-th order zero of the holomorphic function $f$.

\item[\quad] (b)
Define: $z_0$ is an essential singularity of the holomorphic function $F$.

\item[\quad] (c)
Can the poles of a holomorphic function have an accumulation point? Explain.

\item[\quad] (d)
Just what kind of set can be the set of poles of a meromorphic function?

\item[\quad] (e)
$f$ is entire and $f(\Bbb Z) = 0$. Explain why
$\frac{f(z)}{\sin (2\pi z)}$ is an entire function.

\item[10.]
Show that
Re$\left(\frac{e^{it} + z}{e^{it}-z} \right) = \frac{1-|z|^2}{|e^{it}-z|^2}$
and that its integral over $t \in [0,2\pi]$ is $2\pi$ if $z \in \Bbb D$.
What is the value of this integral for
$z \in \Bbb C\ \backslash \overline{\Bbb D}$?





\end{description}    
%\end{Large}
\end{document}














