%Burckel & Bennett, Spring 2002, QE-CA
\documentclass[12pt]{article}
\usepackage{amsxtra,latexsym,amssymb,amsthm}


\setlength{\parindent}{0cm}

\def\ds{\displaystyle}
\def\U{{\mathbb U} }
\def\H{{\mathbb H} }
\def\D{{\mathbb D} }
\def\T{{\mathbb T} }
\def\C{{\mathbb C} }
\def\N{{\mathbb N} }
\def\Z{{\mathbb Z} }
\def\S{{\mathbb S} }
\def\R{{\mathbb R} }

\def\Im{{\hbox{\,Im\,}}}
\def\Re{{\hbox{\,Re\,}}}
\def\csc{{\hbox{\,csc\,}}}
\def\open{{\hbox{\,open\,}}}
\def\grad{{\hbox{\,grad\,}}}
\def\span{{\hbox{\,span\,}}}
\def\id{{\hbox{\,id\,}}}

%\pagestyle{empty}

\textwidth=6.5truein
\textheight=9.5truein
\hoffset=-1truein
\voffset=-1truein

\begin{document}

\def\R{{\mathbb R}}

\begin{large}
\begin{center}
{\bf
  COMPLEX ANALYSIS QUALIFYING EXAM}\\
  January 2002\\
  (Burckel \& Bennett)
\end{center}
\end{large}

\vspace{-.5in}


{
\begin{list}{}
{\itemsep=0em \parsep=0em }\item[]
\end{list}
}

\begin{description}
%\setlength{\itemsep}{0em}\newlength{\parsep}{20em}

\item Do any 7 of the 9 problems. The notation is:
$\N$ for the natural numbers $(1,2,3,\dots)$,
$\R$ the reals,
$\C$ the complex \underbar{plane},
$\D:=\{ z\in\C:|z|<1\}$,
$\overline{\D}:=\{z\in\C:|z|\leq 1\}$.
For open $U\subset\C$, $H(U)$ is the set of all holomorphic
  functions on $U$,
$C(\overline{\D})$ the set of all continuous complex-valued functions on
$\overline\D$.

\item 1.
Let $S$ denote the open sector in the upper half-plane
having bounding rays $\{t\in\R:t\geq 0\}$
and $\{(-\frac{1}{2}+\frac{\sqrt{3}}{2}i)t:t\geq 0\}$
and let $L$ denote the lens-shaped region
$\{z\in\D:|z-1|<2\,\&\, |z+1|<2\}$.
Exhibit a Moebius (=fractional-linear) transformation of $S$ onto $L$
or prove that none exists.

\item 2.
The function $f\in H(\D)$ is defined by $f(z):=\ds\sum^\infty_{n=1} z^n$,
$z\in\D$.
Find the Laurent series for its analytic continuation into the
annulus $\C\backslash\overline{\D}$.

\item 3.
\vspace{.2in}
\begin{tabular}{ll}
\begin{minipage}[c]{1.65in}
The piecewise-smooth closed curve $\Gamma$ is\\
 defined on $[0,4]$ by
\end{minipage}
&
\begin{minipage}[t]{3in}
$\qquad
\Gamma(t):=
\begin{cases} -ie^{2\pi it}   & 0\leq t\leq 1\\
             -i+4(t-1)       & 1\leq t\leq 2\\
             -i+4e^{\pi i(t-2)}& 2\leq t\leq 3\\
             -i+4(t-4)       & 3\leq t\leq 4.
\end{cases}$
\end{minipage}
\end{tabular}
\\~\\
Sketch this curve and compute $\int_\Gamma \csc(z)dz$.

\item 4. (i)
Formulate a Morera theorem in $\D$ that involves only rectangles parallel to
the axes. (A proof is not required.)

\item\quad (ii)
Using (i), outline a proof that ``straight lines are removable for
holomorphic functions", meaning that
$C(\overline{\D})\cap H(\D\backslash \R)\subset H(\D)$.


\item 5.
$U \open\subset \C$, $g:U\to\C$ is continuous and $g^n$ is
holomorphic (for some $n\in\N$). Show that $g$ itself is holomorphic in $U$.
\\
\underbar{Hint}: First consider the case that $g$ is zero-free; work locally.


%\vfill\eject

\item 6.
$f$ is holomorphic and non-constant in the bounded region $\Omega$ and
$\lim_{z\to b}|f(z)|=1$ for each $b$ in the boundary of $\Omega$.

\item\quad (i)
Show that $f(\Omega)\subset \D$.

\item\quad (ii)
Show that $f$ has a zero in $\Omega$.

\item\quad (iii)
Use (ii) to show that $\D\subset f(\Omega)$.
[In summary, $f(\Omega)=\D$.]
\\
\underbar{Hint}: Apply (ii) to $\varphi\circ f$ for various
conformal automorphisms $\varphi$ of $\D$.

\item 7.
Say that $f_n$ \underbar{converges} \underbar{continuously}
to $f$ if $f_n(z_n)\to f(z)$ whenever $z_n\to z$.

\item\quad (i)
Show that for $f_n$, $f\in H(\Omega)$ this concept is equivalent to local
uniform convergence of $f_n$ to $f$.

\item\quad (ii)
Use (i) to show that if $f_n$, $g_n$, $f,g$ are all holomorphic self-maps
of $\Omega$ and $f_n\to f$, $g_n\to g$ (locally uniformly),
then $f_n\circ g_n\to f\circ g$.


\item 8.
$h:\D\to\R$ is harmonic and
$f:=\ds\frac{\partial h}{\partial x}-i\frac{\partial h}{\partial y}$.

\item\quad (i)
Show that $f$ is holomorphic in $\D$.
(You may wish to use the definition ``harmonic $\Leftrightarrow$
Laplacian identically $0$".)

\item\quad (ii)
Show that if $F$ is an antiderivative of $f$,
then $h$ and $\Re F$ differ only by a constant.

\item\quad (iii)
$h:\overline{\D}\to\R$ is continuous and $h$ is harmonic in $\D$.
How would you find a harmonic conjugate $k$ for $h$ in $\D$?

\item 9.
Suppose $f:\C\to\C$ is holomorphic and
$ \Im f(z)>0 \quad \forall z\in\C$. Prove that $f$ is constant.




\end{description}
\end{document}
