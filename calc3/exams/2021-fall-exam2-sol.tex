\documentclass[12pt,answers]{exam}

\usepackage{amsmath,amsfonts,amssymb,mathtools,physics,commath}
\newcommand{\vect}[1]{\left\langle #1\right\rangle}

\pagestyle{headandfoot}
\firstpageheadrule
\runningheadrule
\firstpageheader{Math 222}{Exam 2|Solutions, Page \thepage\ of \numpages}{October 14, 2021}
\runningheader{Math 222}{Exam 2|Solutions, Page \thepage\ of \numpages}{October 14, 2021}
\runningfooter{}{}{}

\title{2021 Fall Calc 3 Exam 2|Solutions}
\author{Winston Cheong}
\date{}

\begin{document}
% \maketitle
\begin{questions}
\question
Answer the following questions concerning the vector valued function
\[
    \vb{r}(t) = \vect{4t^3, \pi \cos(\pi t), \frac{1}{t+1}}
\]
\begin{parts}
\part[5] Evaluate $\vb{r}'(t)$.
\begin{solution}
    \[
    \vb{r} = \vect{12t^2, -\pi^2 \sin(\pi t), -\frac{1}{(t+1)^2}}
    \]
\end{solution}

\part[5] Evaluate 
\[
    \int_0^2 \vb r(t) \dif t.
\]
\begin{solution}
    \begin{align*}
    \int_0^2 \vb r(t) \dif t &= \vect{\eval{t^4}_0^2,\, \eval{\sin(\pi t)}_0^2,\, \eval{\ln|t+1|}_0^2} \\ 
    &= \vect{16, 0, \ln 3}
    \end{align*}
\end{solution}
\end{parts}
\newpage
\question
Consider the vector valued function 
\[
    \vb r(t) = \vect{e^t, 1-2e^t, 2e^t + 1}
\]
for $0 \le t \le \ln(2)$.
\begin{parts}
\part[10] Find the arc-length function $s(t)$ of $\vb{r}(t)$.
\begin{solution}
    We compute $s(t) = \int_1^t \| \vb{r}'(u) \| \dif u$.
    \begin{align*}
        \vb{r}'(t) &= \vect{e^t, -2e^t, 2e^t} \\ 
        \|\vb{r}'(t)\| &= \sqrt{e^{2t}+ 4e^{2t}+ 4e^{2t}} = \sqrt{9e^{2t}} = 3e^t 
    \end{align*}
    Hence
    \[
        s(t) = \int_1^t 3e^u \dif u = \eval{3e^u}_1^t = \boxed{3e^t - 3e}
    \]
\end{solution}
\part[5] Find the arc-length parametrization $\vb{r}(s)$.
\begin{solution}
    Since $s = 3e^t - 3e$ we have $t = \ln(\frac{s+3e}{3})$. Substituting gives
    \[
        \boxed{\vb{r}(s) = \vect{\frac13 s+3e, 1-\frac23(s+3e), \frac23(s+3e) + 1}}
    \]
\end{solution}
\part[5] Recalling that curvature is $\kappa = \|\vb{r}''(s)\|$, find the curvature of the curve paramerized by $\vb{r}(t)$. Explain your answer by identifying the curve.
\begin{solution}
    Taking derivatives with respect to $s$,
    \begin{align*}
        \vb{r}'(s) &= \vect{\frac13, -\frac23, \frac23} \\ 
        \vb{r}''(s) &= \vect{0, 0, 0} \\ 
        \intertext{so}
        \kappa &=  \| \vect{0, 0, 0}\| = \boxed{0}
    \end{align*}
    The curve is a line.
\end{solution}
\end{parts}

\newpage
\question Find the limit, if it exists. If the limit does not exist, explain why.
\begin{parts}
\part[5]
\[
\lim_{(x,y) \to (0,0)} \frac{x^2y - xy}{x^4+y^2x^2}
\]
\begin{solution}
    We simplify before converting to polar:
    \begin{align*}
\lim_{(x,y) \to (0,0)} \frac{x^2y - xy}{x^4+y^2x^2} 
&= \lim_{(x,y) \to (0,0)} \frac{xy(x - 1)}{x^2(x^2+y^2)}  \\ 
&= \lim_{r\to0} \frac{r^2 \cos \theta \sin \theta (r\cos \theta - 1)}{r^4 \cos^2 \theta} \\ 
&= \lim_{r\to0} \frac{\sin \theta (r\cos \theta - 1)}{r^2 \cos \theta} \\ 
    \end{align*}
    This funciton still depends on $\theta$, so the limit \fbox{does not exist}
\end{solution}
\part[5]
\[
\lim_{(x,y) \to (\pi/4,\pi/4)} \frac{\cos(x) - \sin(y)}{\cos^2(x) - \sin^2(y)}
\]
\begin{solution}
\begin{align*}
\lim_{(x,y) \to (\pi/4,\pi/4)} \frac{\cos(x) - \sin(y)}{\cos^2(x) - \sin^2(y)}
&= \lim_{(x,y) \to (\pi/4,\pi/4)} \frac{1}{\cos x + \sin y} \\ 
&= \frac{1}{\cos \frac\pi4 + \sin \frac\pi4} = \boxed{\frac{1}{\sqrt{2}}}
\end{align*}
\end{solution}
\part[5]
\[
\lim_{(x,y) \to (0,0)} \frac{x^2y^2}{x^4 + x^2y^2 + y^4}
\]
\begin{solution}
Approaching along $y = 0$, the limit reduces to $\displaystyle \lim_{x\to0} \frac{0}{x^4} = 0$. \\ 
Approaching along $x= y$, the limit reduces to $\displaystyle \lim_{x \to 0} \frac{x^4}{3x^4} = \frac13$.

Thus the limit \fbox{does not exist}
\end{solution}
\end{parts}

\newpage
\question
Evaluate the partial derivatives, if they exist. If they do not exist, explain why
\begin{parts}
\part[5] $f_y(3, 6)$ for $f(x, y) = xe^{y-6} + \ln(xy)$.
\begin{solution}
    \begin{align*}
        f_y &= xe^{y-6} + \frac{1}{xy} x \\ 
            &= xe^{y-6} + \frac{1}{y} \\ 
        f_y(3,6) &= \boxed{3e^3 + \frac 16}
    \end{align*}
\end{solution}
\part[5]
$\frac{\partial^3 f}{\partial x\partial y\partial z}(1,2,3)$ for $f(x,y,z) = xy+yz+xz+xyz$.
\begin{solution}
    The only term with all three variables is the last. The other terms will become 0 when the appropriate partial derivative is taken. 
    We only need to concern ourselves with 
    \[
        \frac{\partial^3}{\partial x \partial y \partial z} (xyz) = \boxed{1}
    \]
\end{solution}
\part[5]
$f_z(0,3,0)$ for 
\[
f(x,y,z) = y + \sqrt{xyz + x^2 + z^2}
\]
\begin{solution}
    \[
        f_z = \frac{xy+2z}{2\sqrt{xyz+x^2+z^2}}
    \]
    Evaluating at $(0, 3, 0)$ gives $\frac 00$ hence $f_z(0,3,0)$ \fbox{does not exist}
\end{solution}
\end{parts}

\newpage
\question
Consider the function $f(x, y) = xy - 2x + y^2$.
\begin{parts}
\part[5] Give the linearization $L(x, y)$ of $f(x, y)$ at $(-1, 1)$.
\begin{solution}
    The linearization centered at $(-1, 1)$ is 
    \[
        L(x, y) = f(-1, 1) + f_x(-1, 1)(x+1) + f_y(-1, 1)(y-1)
    \]
    Some side calculation gives
    \begin{align*}
        f(-1, 1) &= 2 \\ 
        f_x(x, y) &= y -2 \\ 
        f_x(-1, 1) &= -1 \\ 
        f_y(x, y) &= x + 2y \\ 
        f_y(-1, 1) &= 1
    \end{align*}
    so the linearization is 
    \begin{align*}
        L(x, y) &= f(-1, 1) + f_x(-1, 1)(x+1) + f_y(-1, 1)(y-1) \\ 
        &= 2 -(x+1) + (y-1) \\ 
        \Aboxed{L(x, y) &= -x+y}
    \end{align*}
\end{solution}
\part[5] Use your result from part (a) to write the equation of the tangent plane to the graph of $f(x, y)$ at $(-1, 1)$.
\begin{solution}
    \[
        z = -x + y 
    \]
\end{solution}
\end{parts}

\newpage
\question
Let 
\[
f(x, y, z) = xy + yz + xz + 1
\]
\begin{parts}
\part[5] Find the gradient of $f$.
\begin{solution}
\[
    \nabla f = \vect{y+z, x+z, y+x}
\]
\end{solution}
\part[5] Let $\vb{u} = \vect{\frac{\sqrt 3}{3}, \frac{\sqrt 3}{3},\frac{\sqrt 3}{3}}$ and find the directional derivative $D_{\vb{u}} f(\sqrt 3, \sqrt 3, \sqrt 3)$.
\begin{solution}
\begin{align*}
    D_{\vb{u}} f(\sqrt 3, \sqrt 3, \sqrt 3) 
    &= \nabla f(\sqrt 3, \sqrt 3, \sqrt 3) \vdot \vb u \\ 
    &= \vect{2\sqrt 3, 2 \sqrt 3, 2\sqrt 3} \vdot \vect{\frac{\sqrt 3}{3}, \frac{\sqrt 3}{3},\frac{\sqrt 3}{3}} \\ 
    &= 2 \sqrt 3 \frac{\sqrt 3}{3} \cdot 3 = \boxed{6}
\end{align*}
\end{solution}
\part[5] Give an example of a vector $\vb{v}$ for which $f$ is increasing in the direction of $\vb{v}$ starting at $(-1, 1, 1)$.
\begin{solution}
\[
    \nabla f(-1, 1, 1) = \boxed{\vect{2, 0, 0} =: \vb v}
\]
\end{solution}
\end{parts}

\newpage
\question Consider the function
\[
    f(x, y, z) = x^2 + y^2 + z
\]
\begin{parts}
    \part[10] Use the chain rule to calculate $\frac{\partial f}{\partial \varphi}$ at the point $(\rho, \theta, \phi) = (1, 0, \pi/2)$ in spherical coordinates.
    \begin{solution}
        \begin{align*}
            \frac{\partial f}{\partial \varphi}
             & = \pdv{f}{x}\pdv{x}{\varphi} + \pdv{f}{y} \pdv{y}{\varphi} + \pdv{f}{z} \pdv{z}{\varphi}                        \\
             & = 2x \cdot \rho \cos \theta \cos \varphi + 2 y \cdot \rho \sin \theta \cos \varphi + 1 \cdot -\rho \sin \varphi \\
             & = 0 + 0 + -1 = \boxed{-1}
        \end{align*}
    \end{solution}
    \part[5] Confirm your result by expressing $f$ in spherical coordinates and taking its partial derivative with respect to $\varphi$.
    \begin{solution}
        \begin{align*}
            f                      & = \rho^2 \cos^2 \theta \sin^2 \varphi + \rho^2 \sin^2 \theta \sin^2 \varphi + \rho \cos \varphi \\
                                   & = \rho^2 \sin^2 \varphi + \rho \cos \varphi                                                     \\
            f_\varphi              & = \rho^2 \cdot 2 \sin \varphi \cos \varphi - \rho \sin \varphi                                  \\
            f_\varphi(1, 0, \pi/2) & = 1 \cdot 2 \sin \pi/2 \cos\pi/2 - 1 \cdot \sin \pi/2                                           \\
                                   & = 0 - 1 \cdot 1 = \boxed{-1}
        \end{align*}
    \end{solution}
\end{parts}
\end{questions}

\end{document}