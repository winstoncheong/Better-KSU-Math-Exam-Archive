\documentclass[answers]{exam}
\usepackage{amsmath,amsfonts,amssymb,commath,mathtools,physics}
\newcommand{\vect}[1]{\left\langle #1 \right\rangle}
\newcommand{\tcurl}{\operatorname{curl}}

\title{2019 Spring Calc 3 Final|Solutions}
\author{Winston Cheong}
\date{}
\begin{document}
\maketitle

\begin{questions}
\question
Let $\vb{u} = \vect{1, 1, 0}$, $\vb{v} = \vect{1, 0, 1}$ and $\vb{w} = \vect{0, 1, 1}$.
\begin{parts}
\part[5] Compute the volume of the parallelopiped spanned by $\vb u$, $\vb v$ and $\vb{w}$. 
\begin{solution}
\[
    \mqty| 1 & 1 & 0 \\ 1 & 0 & 1 \\ 0 & 1 & 1| = 1 \cdot \mqty|0 & 1 \\ 1 & 1| - 1 \cdot \mqty|1 & 1 \\ 0 & 1| + 0 \cdot \mqty|1 & 0 \\ 0 & 1|
    = -1 -1 + 0 = -2
\]
So the volume is \boxed{2}.
\end{solution}
\part[5] Compute the angle $\theta$ between $\vb v$ and $\vb w$.
\begin{solution}
    \[
        \cos \theta = \frac{\vb{u} \vdot \vb{w}}{\|\vb{v}\|\, \|\vb{w}\|} = \frac{1}{\sqrt 2\cdot \sqrt 2}
    \]
    Thus $\theta = \cos[-1](\frac12) = \boxed{\frac\pi3}$
\end{solution}
\part[5] Give the equation for the plane parallel to $\vb u$ and $\vb v$ and passing through the origin.
\begin{solution}
    The normal vector for the plane is 
    \[
    \vb{n} = \vb{u} \cross \vb{v} = \mqty| \vb{i} & \vb{j} & \vb{k} \\ 1 & 1 & 0 \\ 1 & 0 & 1| = \vect{1, -1, -1}
    \]
    Thus the equation for the plane is
    \[
        \boxed{\vect{1, -1, -1} \vdot \vect{x, y, z} = 0}
        \quad \text{or} \quad \boxed{x-y-z = 0}
    \]
\end{solution}
\end{parts}
\question Consider the curve $\mathcal{C}$ given by the parametrization 
\[
    \vb{r}(t) = \vect{\sin(t), \cos(t), e^t}\quad \text{for } 0\le t \le \pi
\]
\begin{parts}
\part[5] Find the speed of $\vb{r}(t)$ as a function of $t$.
\begin{solution}
    \[
    s(t) = \|\vb{r}'(t)\| = \| \vect{\cos t, -\sin t, e^t} \| = \sqrt{\cos^2 t + \sin^2 t + e^{2t}}
    = \boxed{\sqrt{1 + e^{2t}}}
    \]
\end{solution}
\part[5] Compute the scalar line integral 
\[
    \int_{\mathcal C} 3x^2 z^2 + 3y^2 z^2 \dif s.
\]
\begin{solution}
    Formula sheet:
    \[
        \int_{\mathcal C} f(x, y, z) \dif s = \int_a^b f(\vb{r}(t)) \|\vb{r}'(t)\| \dif t
    \]
    So
    \begin{align*}
    \int_{\mathcal C} 3x^2 z^2 + 3y^2 z^2 \dif s 
    &= \int_0^\pi (3\sin^2 t e^{2t} + 3\cos^2 t e^{2t}) \sqrt{1+e^{2t}} \dif t \\ 
    &= \int_0^\pi 3e^{2t} \sqrt{1+e^{2t}} \dif t
    \qquad
    \text{u sub:}
    \left(
    \begin{aligned}
   u &= 1+e^{2t} \\ 
   \dif u &= 2e^{2t} \dif t \implies \dif t = \frac{\dif u}{2e^{2t}}
    \end{aligned} 
    \right)
    \\
    &= \int 3 e^{2t} \cdot u^{1/2} \frac{\dif u}{2e^{2t}} \\ 
    &= \frac32 \int u^{1/2} \dif u \\ 
    &= u^{3/2} = \left[(1+e^{2t})^{3/2}\right]_0^\pi
    = \boxed{(1+e^{2\pi})^{3/2} - 2^{3/2}}
    \end{align*}


\end{solution}
\end{parts}

\question
Calculate the following quantities if they exist. Otherwise, explain why they do not exist. Justify either response.
\begin{parts}
\part[5] 
\[
    \lim_{(x, y) \to (0, 0)} \frac{x^2y^2}{x^2+y^2}
\]
\part[5]
For 
\[
    f(x, y, z) = \cosine(z^2 - y^2) + y \sin(x)
\]
compute 
\[
    f_{xy}(x,y,z)
\]
\part[5]
For $f(x,y,z) = x + y^2 + z^3$ find the change in $f(x,y,z)$ as one moves in the direction of the unit vector $\vb{u} = \frac{\sqrt 3}{3} \vect{1,1,1}$ starting at $\vect{2,0,-1}$.
\part[5] Find the equation for the tangent plane to the surface 
\[
    z = xy
\]
at the point $(1,2,2)$.
\end{parts}
\question
Let 
\[
    f(x,y) = x^3-12x+y^2
\]
and $\mathcal{D}$ be the square $[-3,3]\times[-3,3]$.
\begin{parts}
\part[5] Find the critical points of $f(x, y)$ in the interior of $\mathcal{D}$.
\part[5] Describe the local behavior of $f(x, y)$ at the critical points found in part (a).
\part[5] Find the maximum value of $f$ on $\mathcal D$.
\end{parts}

\question[10]
Let $\mathcal W = [0, 1] \times [-1, 0] \times [0, 2]$. 
Evaluate the triple integral
\[
    \iiint_{\mathcal W} (2x + z)e^y \dif V
\]
\question Evaluate the following integrals.
\begin{parts}
\part[10] Let $\mathcal D$ be the region $x^2 + y^2 \le 4$, $0 \le y$, $x \le 0$. Evaluate
\[
    \iint_{\mathcal D} 3x \dif A.
\]
\part[10] Let $\mathcal D$ be the region between the lines $y = -x$, $y = -1$ and $x = -1$.
Compute the integral 
\[
    \iint_{\mathcal D} 2y \dif A.
\]
\end{parts}

\question Let 
\[
    \vb F = \vect{2x+yz, xz, xy}.
\]
\begin{parts}
\part[5] If $\vb F$ is a conservative vector field, find a potential. Otherwise, explain why it is not conservative.
\part[5] Let $\mathcal C$ be the oriented curve with parametrization
\[
    \vb r(t) = \vect{\sin[6](\pi t)+t+1, e^t + e^{-t}, e^{t^2-1} - 1}
\]
for $-1 \le t \le 1$.
Compute 
\[
    \int_{\mathcal C} \vb F \vdot \dif \vb r.
\]
\part[5] Is there a vector potential for $\vb{F}$ (a vector field $\vb{A}$ that satisfies $\vb{F} = \operatorname{curl}(\vb{A})$)? 
Explain your response.
\part[5] Let $\mathcal S$ be the sphere $x^2 + y^2 + z^2 = 9$ oriented outwardly.
Compute the surface integral
\[
    \iint_{\mathcal S} \vb{F} \vdot \dif \vb{S}.
\]
State any theorems used in the computation.
\end{parts}

\question
Let $\mathcal D$ be the lower half disc
\[
    \mathcal D = \{(x, y) : x^2 + y^2 \le 1,\, y \le 0\}.
\]
The boundary of $\mathcal D$ consists of the line segment $\mathcal C_1$ along the $x$-axis oriented from $(1,0)$ to $(-1, 0)$ and the semi-circle 
\[
    \mathcal C_2 = \{(x, y) : y = -\sqrt{1-x^2},\, -1 \le x \le 1\}
\]
oriented counter-clockwise. Let $\vb F$ be the vector field 
\[
    \vb{F} = \vect{-yx^2, xy^2}.
\]
\begin{parts}
\part[5]
Using polar coordinates, calculate the double integral
\[
    \iint_{\mathcal D} x^2 + y^2 \dif A
\]
\part[5] Compute the line integral
\[
    \int_{\mathcal C_1} \vb{F} \vdot \dif \vb{r}
\]
\part[5] Using only Green's Theorem and the computations in parts (a) and (b), compute the vector line integral 
\[
    \int_{\mathcal C_2} \vb{F} \vdot \dif \vb{r}
\]
\end{parts}
\question
Let $\mathcal S$ be the cylinder
\[
    \{(x, y, z) : x^2 + y^2 = 1,\, 0 \le z \le 3\}
\]
oriented outward and $\vb{F} = \vect{zy, -zx, 0}$.
\begin{parts}
\part[5] Compute $\tcurl(\vb F)$.
\part[5] Calculate
\[
    \iint_{\mathcal S} \tcurl(\vb{F}) \vdot \dif \vb{S}
\]
\part[5] The boundary of $\mathcal S$ consists of a unit circle $\mathcal C_1$ on the $xy$-plane oriented counterclockwise and a unit circle $\mathcal C_2$ on the $z=3$ plane oriented clockwise.
Noting that the vector field is zero on the $xy$-plane, one easily sees that 
\[
    \int_{\mathcal C_1} \vb{F} \vdot \dif \vb{r} = 0
\]
Using only this fact, Stokes' Theorem and your result from part (b), compute the vector line integral
\[
    \int_{\mathcal C_2} \vb{F} \vdot \dif \vb{r}
\]
\end{parts}

\end{questions}

\end{document}